\documentclass[12pt]{amsart}
\usepackage{fullpage,amssymb}
\pagestyle{empty}
\title{The MPFR Library: Algorithms and Proofs}
\author{The MPFR team}
\date{\tt www.mpfr.org}
\def\O{{\mathcal O}}
\def\n{\textnormal}
\def\pinf{\bigtriangleup}
\def\minf{\bigtriangledown}
\def\q{\hspace*{5mm}}
\def\ulp{{\rm ulp}}
\def\Exp{{\rm EXP}}
\def\prec{{\rm prec}}
\def\sign{{\rm sign}}
\def\Paragraph#1{\noindent {\sc #1}}
\def\Z{{\mathcal Z}}
\def\N{{\mathcal N}}
\def\If{{\bf if}}
\def\then{{\bf then}}
\def\Else{{\bf else}}
\def\elif{{\bf elif}}
\def\for{{\bf for}}
\def\to{{\bf to}}
\def\while{{\bf while}}
\def\err{{\rm err}}

\newcommand{\U}[1]{\quad \mbox{[Rule~\ref{#1}]}}

\newtheorem{Rule}{Rule}

\begin{document}
\maketitle

\tableofcontents

\section{Basic rules}

In the following, $n$ is the precision (number of bits of the mantissa),
each floating-point number is written $x = m \cdot 2^e$
with $\frac{1}{2} \le |m| < 1$ and $e := {\rm EXP}(x)$, and
$\ulp(x) := 2^{{\rm EXP}(x) - n}$.

\begin{Rule} \label{R1}
$2^{-n} |x| < \ulp(x) \le 2^{-n+1} |x|$.
\end{Rule}
\begin{proof}
Obvious from $x = m \cdot 2^e$ with $\frac{1}{2} \le |m| < 1$.
\end{proof}

\begin{Rule} \label{R2}
If $a$ and $b$ have same precision $n$,
and $|a| \le |b|$, then $\ulp(a) \le \ulp(b)$.
\end{Rule}
\begin{proof}
Write $a = m_a \cdot 2^{e_a}$ and $b = m_b \cdot 2^{e_b}$.
Then $|a| \le |b|$ implies $e_a \le e_b$, thus
$\ulp(a) = 2^{e_a-n} \le 2^{e_b-n} = \ulp(b)$.
\end{proof}

\begin{Rule} \label{R3}
Let $\ulp(x) := 2^{{\rm EXP}(x) - n}$, then
$2^{n-1} \ulp(x) \le |x| < 2^{n} \ulp(x)$.
\end{Rule}
\begin{proof}
Obvious from $x = m \cdot 2^e$ with $\frac{1}{2} \le |m| < 1$.
\end{proof}

\begin{Rule} \label{R4}
$\frac{1}{2} |a| \cdot \ulp(b) < \ulp(a b) < 2 |a| \cdot \ulp(b)$.
\end{Rule}
\begin{proof}
Write $a = m_a 2^{e_a}$, $b = m_b \cdot 2^{e_b}$, and $a b = m 2^e$
with $\frac{1}{2} \le m_a, m_b, m < 1$,
then $\frac{1}{4} 2^{e_a+e_b} \le |a b| < 2^{e_a+e_b}$,
thus $e = e_a + e_b$ or $e = e_a + e_b - 1$, which implies
$\frac{1}{2} 2^{e_a} \ulp(b) \le \ulp(a b) \le 2^{e_a} \ulp(b)$
using $2^{e_b-n} = \ulp(b)$, and the rule follows from
the fact that $|a| < 2^{e_a} \le 2|a|$ (equality on the right side can
occur only if $e = e_a + e_b$ and $m_a = \frac{1}{2}$, which are
incompatible).
\end{proof}

\begin{Rule} \label{R5}
$\ulp(2^k a) = 2^k \ulp(a)$.
\end{Rule}
\begin{proof}
Easy: if $a = m_a \cdot 2^{e_a}$, then $2^k a = m_a \cdot 2^{e_a+k}$.
\end{proof}

\begin{Rule} \label{R6}
Let $x > 0$, $o(\cdot)$ be any rounding, and $u := o(x)$, then $\frac{1}{2} u
	< x < 2 u$.
\end{Rule}
\begin{proof}
Assume $x \geq 2 u$, then $2u$ is another representable number which is closer 
from $x$ than $u$, which leads to a contradiction. The same argument proves
$\frac{1}{2} u < x$.
\end{proof}

\begin{Rule} \label{R7}
$\frac{1}{2} |a| \cdot \ulp(1) \leq \ulp(a) \leq |a| \cdot \ulp(1)$.
\end{Rule}
\begin{proof}
The left inegality comes from Rule~\ref{R4} with $b=1$,
and the right one from $|a| \ulp(1) \geq \frac{1}{2} 2^{e_a} 2^{1-n} =\ulp(a)$.
\end{proof}

\begin{Rule} \label{R8}
For any $x \neq 0$ and any rounding mode $o(\cdot)$,
we have $\ulp(x) \leq \ulp(o(x))$, and equality holds when rounding towards
zero, towards $-\infty$ for $x>0$, or towards $+\infty$ for $x<0$.
\end{Rule}
\begin{proof}
Without loss of generality, assume $x > 0$.
Let $x = m \cdot 2^e$ with $\frac{1}{2} \leq m < 1$.
As $\frac{1}{2} 2^{e_x}$ is a machine number, necessarily $o(x) \geq
\frac{1}{2} 2^{e_x}$, thus by Rule~\ref{R2}, then $\ulp(o(x)) \geq
2^{e_x - n} = \ulp(x)$.
If we round towards zero, then $o(x) \leq x$ and by Rule~\ref{R2} again,
$\ulp(o(x)) \leq \ulp(x)$.
\end{proof}

\begin{Rule} \label{R9}
\begin{eqnarray}\nonumber
&&\n{For}\;\;  error(u) \leq k_u \ulp(u),\;\; u.c_u^- \leq x \leq u.c_u^+)\\\nonumber
&&\n{with}\;\;   c_u^{-}=1- k_u 2^{1-p} \n{ and } c_u^{+}=1+ k_u 2^{1-p}
\end{eqnarray}

\begin{eqnarray}\nonumber
&&\n{For}\;\;  u=o(x),\;\; u.c_u^- \leq x \leq u.c_u^+\\\nonumber
&&\n{if}\;\;  u=\pinf(x),\n{ then } c_u^+=1\\\nonumber
&&\n{if}\;\;  u=\minf(x),\n{ then } c_u^-=1\\\nonumber
&&\n{if}\;\;  \n{for $x<0$ and } u=Z(x),\n{ then } c_u^+=1 \\\nonumber
&&\n{if}\;\;  \n{for $x>0$ and } u=Z(x),\n{ then } c_u^-=1 \\\nonumber
&&\n{else}\;\;   c_u^{-}=1-2^{1-p} \n{ and } c_u^{+}=1+2^{1-p}
\end{eqnarray}
\end{Rule}

\section{Low level functions}

\subsection{The {\tt mpfr\_add} function}

\begin{verbatim}
   mpfr_add (A, B, C, rnd)
   /* on suppose B et C de me^me signe, et EXP(B) >= EXP(C) */

   0. d = EXP(B) - EXP(C) /* d >= 0 par hypothe`se */
   1. Soient B1 les prec(A) premiers bits de B, et B0 le reste
             C1 les bits de C correspondant a` B1, C0 le reste
   /* B0, C1, C0 peuvent e^tre vides, mais pas B1 */

          <----------- A ---------->
          <----------- B1 ---------><------ B0 ----->
             <---------- C1 -------><------------ C0 ----------->

   2. A <- B1 + (C1 >> d)
   3. q <- compute_carry (B0, C0, rnd)
   4. A <- A + q
\end{verbatim}

\subsection{The {\tt mpfr\_cmp2} function}

This function computes the exponent shift when subtracting $c > 0$ from
$b \ge c$. In other terms, if ${\rm EXP}(x) := 
\lfloor \frac{\log b}{\log 2} \rfloor$, it returns:
it returns ${\rm EXP}(b) - {\rm EXP}(b-c)$.

This function admits the following specification in terms of the binary
representation of the mantissa of $b$ and $c$: if $b = u 1 0^n r$ and
$c = u 0 1^n s$, where $u$ is the longest common prefix to $b$ and $c$,
and $(r,s)$ do not start with $(0, 1)$, then ${\tt mpfr\_cmp2}(b,c)$ returns
$|u| + n$ if $r \ge s$, and $|u| + n + 1$ otherwise, where $|u|$ is the number
of bits of $u$.

As it is not very efficient to compare $b$ and $c$ bit-per-bit, we propose
the following algorithm, which compares $b$ and $c$ word-per-word.
Here $b[n]$ represents the $n$th word from the mantissa of $b$, starting from
the most significant word $b[0]$, which has its most significant bit set.
The values $c[n]$ represent the words of $c$, after a possible shift if the
exponent of $c$ is smaller than that of $b$.
\begin{verbatim}
   n = 0; res = 0;
   while (b[n] == c[n])
      n++;
      res += BITS_PER_MP_LIMB;

   /* now b[n] > c[n] and the first res bits coincide */

   dif = b[n] - c[n];
   while (dif == 1)
      n++;
      dif = (dif << BITS_PER_MP_LIMB) + b[n] - c[n];
      res += BITS_PER_MP_LIMB;

   /* now dif > 1 */

   res += equal_leading_bits(b[n], c[n]);

   if (!is_power_of_two(dif))
      return res;

   /* otherwise result is res + (low(b) < low(c)) */
   do
      n++;
   while (b[n] == c[n]);
   return res + (b[n] < c[n]);
\end{verbatim}

\subsection{The {\tt mpfr\_sub} function}

The algorithm used is as follows, where $w$ denotes the number of bits
per word. We assume that $a$, $b$ and $c$ denote different variables
(if $a:=b$ or $a:=c$, we have first to copy $b$ or $c$), and
that the rounding mode is either $\N$ (nearest),
$\Z$ (towards zero), or $\infty$ (away from zero).
\begin{quote}
Algorithm {\tt mpfr\_sub}. \\
Input: $b$, $c$ of opposite sign with $b > c > 0$, a rounding mode 
$\circ \in \{ \N, \Z, \infty \}$ \\
Side effect: store in $a$ the value of $\circ(b - c)$ \\
Output: $0$ if $\circ(b - c) = b-c$, $1$ if $\circ(b - c) > b-c$,
	and $-1$ if $\circ(b - c) < b-c$ \\
${\tt an} \leftarrow \lceil \frac{\prec(a)}{w} \rceil$, 
${\tt bn} \leftarrow \lceil \frac{\prec(b)}{w} \rceil$, 
${\tt cn} \leftarrow \lceil \frac{\prec(c)}{w} \rceil$ \\
${\tt cancel} \leftarrow {\tt mpfr\_cmp2}(b, c)$; \quad
	${\tt diff\_exp} \leftarrow \Exp(b)-\Exp(c)$ \\
${\tt shift_b} \leftarrow (-{\tt cancel}) \bmod w$; \quad
	${\tt cancel_b} \leftarrow ({\tt cancel} + {\tt shift_b})/w$ \\
\If\ ${\tt shift_b} > 0$ \then\
	${\tt b}[0 \dots \mbox{\tt bn}] \leftarrow
	{\tt mpn\_rshift}({\tt b}[0 \dots {\tt bn}-1], {\tt shift_b})$;
	${\tt bn} \leftarrow {\tt bn} + 1$ \\
${\tt shift_c} \leftarrow ({\tt diff\_exp}-{\tt cancel}) \bmod w$; \quad
${\tt cancel_c} \leftarrow ({\tt cancel} + {\tt shift_c}-{\tt diff\_exp})/w$ \\
\If\ ${\tt shift_c} > 0$ \then\
	${\tt c}[0 \dots \mbox{\tt cn}] \leftarrow
	{\tt mpn\_rshift}({\tt c}[0 \dots {\tt cn}-1], {\tt shift_c})$;
	${\tt cn} \leftarrow {\tt cn} + 1$ \\
$\Exp(a) \leftarrow \Exp(b) - {\tt cancel}$; \quad
	$\sign(a) \leftarrow \sign(b)$ \\
$a[0 \dots {\tt an}-1] \leftarrow b[{\tt bn} - {\tt cancel_b} - {\tt an}
	\dots {\tt bn} - {\tt cancel_b} - 1]$ \\
$a[0 \dots {\tt an}-1] \leftarrow a[0 \dots {\tt an}-1] - c[{\tt cn} -
	{\tt cancel_c} - {\tt an} \dots {\tt cn} - {\tt cancel_c} - 1]$ \\
${\tt sh} \leftarrow {\tt an} \cdot w - \prec(a)$; \quad
	$r \leftarrow a[0] \bmod 2^{\tt sh}$; \quad
	$a[0] \leftarrow a[0] - r$ \\
\If\ $\circ = \N$ and ${\tt sh} > 0$ \then \\
\q \If\ $r > 2^{{\tt sh}-1}$ \then\
	$a \leftarrow a + \ulp(a)$; return $1$
   \elif\ $0 < r < 2^{{\tt sh}-1}$ \then\ return $-1$ \\
\elif\ $\circ \in \{ \Z, \infty \}$ and $r > 0$ \then \\
\q \If\ $\circ = \Z$ return $-1$
   \Else\ $a \leftarrow a + \ulp(a)$; return $1$ \\
${\tt bl} \leftarrow {\tt bn} - {\tt an} - {\tt cancel_b}$ \\
${\tt cl} \leftarrow {\tt cn} - {\tt an} - {\tt cancel_c}$ \\
\for\ $k=0$ \while\ ${\tt bl} > 0$ and ${\tt cl} > 0$ {\bf do} \\
\q ${\tt bl} \leftarrow {\tt bl} - 1$; ${\tt bp} \leftarrow b[{\tt bl}]$ \\
\q ${\tt cl} \leftarrow {\tt cl} - 1$; ${\tt cp} \leftarrow c[{\tt cl}]$ \\
\q \If\ $\circ = \N$ and $k=0$ and ${\tt sh}=0$ \then \\
\q \q \If\ ${\tt cp} \ge 2^{w-1}$ \then\ return $-1$ \\
\q \q $r \leftarrow {\tt bp} - {\tt cp}$; \quad 
	${\tt cp} \leftarrow {\tt cp} + 2^{w-1}$ \\
\q \If\ ${\tt bp} < {\tt cp}$ \then \\
\q \q \If\ $\circ = \Z$ \then\ $a \leftarrow a - \ulp(a)$; \quad
      \If\ $\circ = \infty$ \then\ return $1$ \Else\ return $-1$
\q \If\ ${\tt bp} > {\tt cp}$ \then \\
\q \q \If\ $\circ = \Z$ \then\ return $-1$ \Else\
	$a \leftarrow a + \ulp(a)$; return $1$ \\
\If\ $\circ = \N$ and $r > 0$ \then \\
\q \If\ $a[0] \, {\rm div} \, 2^{\tt sh}$ is odd \then\
	$a \leftarrow a + \ulp(a)$; return $1$ \Else\ return $-1$ \\
Return $0$.
\end{quote}
where $b[i]$ and $c[i]$ is meant as $0$ for negative $i$,
and $c[i]$ is meant as $0$ for $i \ge {\tt cn}$
(${\tt cancel_b} \ge 0$, but ${\tt cancel_c}$ may be negative).

\subsection{The {\tt mpfr\_div} function}
The goals of the code of the {\tt mpfr\_div} function include the fact
that the complexity should, as much as possible while preserving exact
rounding, depend on the precision required on the result rather than
on the precision given on the operands. Let $u = u_n 2^{u_e}$, $v = v_n
2^{v_e}$, where $u_n$ and $v_n$ are in $[1/2, 1[$. Let $q_p$ be the 
precision required on $q$. Put 
$b_p = \min(v_p, q_p + \varepsilon_p)$, 
$a_p = b_p + q_p + \varepsilon_p$, where $\varepsilon_p$ is a small value
to be chosen.

First, a integer division of $u_{hi} = \lfloor 
u_n 2^{a_p} \rfloor$ by 
$v_{hi} = \lfloor v_n 2^{b_p} \rfloor$ 
is performed. Write $u_{hi} = \tilde{q} v_{hi} + \tilde{r}$. 

If this division is not a full one, to obtain the real value 
of the quotient, if $\delta = max(u_p, v_p)$, we have to 
divide $u_n 2^{q_p + \varepsilon_p + \delta}$ by 
$v_n 2^{\delta}$. 

In that case, $2^{q_p + \varepsilon_p + \delta} u_n = \tilde{q}v_n
2^{\delta} + \tilde{r} 2^{\delta - q_p - \varepsilon_p} + u_{lo} -
\tilde{q}v_{lo}$, with obvious notations.

A positive correction on $q$ has to come from the contribution of 
$\tilde{r} 2^{\delta - q_p - \varepsilon_p} + u_{lo}$. The
first term is at most $v_{hi} 2^{\delta - q_p - \varepsilon_p}$.
As for $u_{lo}$, we have $u_{lo} < 2^{\delta-q_p-\varepsilon_p}$. Hence, 
the sum $u_{lo} + \tilde{r} 2^{\delta - q_p - \varepsilon_p} < 2v$, 
and the positive correction is at most 1. 

We now have to estimate $\tilde{q}v_{lo}$. It is easily seen that 
$\tilde{q} < 2^{q_p + \varepsilon_p + 1}$. As for $v_{lo}$, we have
$v_{lo} < 2^{\delta - q_p - \varepsilon_p}$, so that 
$\tilde{q} v_{lo} < 2^{\delta + 1}$, to be compared with $v_n 2^{\delta}$, 
so that a negative correction is at most 3. As a consequence, to be able
to decide rounding after the first stage, one should choose 
$\varepsilon_p \geq 3$ (to include the round-to-nearest case). 

\section{Mathematical constants}

\subsection{Euler's constant}

Euler's constant is computed using the formula $\gamma = S(n) - R(n) - \log n$
where:
\[ S(n) = \sum_{k=1}^{\infty} \frac{n^k (-1)^{k-1}}{k! k}, \quad
   R(n) = \int_n^{\infty} \frac{\exp(-u)}{u} du \sim \frac{\exp(-n)}{n}
        \sum_{k=0}^{\infty} \frac{k!}{(-n)^k}. \]
This identity is attributed to Sweeney by Brent \cite{Brent78}.
We have $S(n) = _2 F_2(1,1;2,2;-n)$ and $R(n) = {\rm Ei}(1, n)$.

\Paragraph{Evaluation of $S(n)$.}
As in \cite{Brent78}, let $\alpha \sim 4.319136566$ the positive root
of $\alpha + 2 = \alpha \log \alpha$, and $N = \lceil \alpha n \rceil$.
We approximate $S(n)$ by
\[ S'(n) = \sum_{k=1}^{N} \frac{n^k (-1)^{k-1}}{k! k}. \]
%         = \frac{1}{N!} \sum_{k=1}^N \frac{a_k}{k}, 
% where $a_k = n^k (-1)^{k-1} N!/k!$ is an integer.
% Therefore $a_k$ is exactly computed, and when dividing it by $k$
% (integer division)
% the error is at most $1$, and thus the absolute error on $S'(n)$ is
% at most $N/N! = 1/(N-1)!$.
The remainder term $S(n) - S'(n)$ is bounded by:
\[ |S(n) - S'(n)| \le \sum_{k=N+1}^{\infty} \frac{n^k}{k!}. \]
Since $k! \ge (k/e)^k$, and $k \ge N+1 \ge \alpha n$, we have:
\[ |S(n) - S'(n)| \le \sum_{k=N+1}^{\infty} \left( \frac{n e}{k} \right)^k
                  \le \sum_{k=N+1}^{\infty} \left( \frac{e}{\alpha} \right)^k
                  \le 2 \left( \frac{e}{\alpha} \right)^N
                  \le 2 e^{-2n} \]
since $(e/\alpha)^{\alpha} = e^{-2}$.

To approximate $S'(n)$, we use the following algorithm, where $m$ is the
working precision, and $a, s, t$ are integer variables:
\begin{quote}
$a \leftarrow 2^m$ \\
$s \leftarrow 0$ \\
{\bf for} $k$ {\bf from} $1$ {\bf to} $N$ {\bf do} \\
\q $a \leftarrow \lfloor \frac{n a}{k} \rfloor$ \\
\q $t \leftarrow \lfloor \frac{a}{k} \rfloor$ \\
\q $s \leftarrow s + (-1)^{k-1} t$ \\
return $x = s/2^m$
\end{quote}
The absolute error $\epsilon_k$ on $a$ at step $k$ satisfies
$\epsilon_k \le 1 + n/k \epsilon_{k-1}$ with $\epsilon_0=0$.
The maximum error is $\epsilon_n \le \frac{n^n}{n!} \le e^n$.
Thus the error on $t$ at step $k$ is less than $1 + e^n/k$,
and the total error on $s$ is bounded by $N (e^n + 1)$.
Hence to get a precision of $n$ bits, we need to use
$m ge n (1 + \frac{1}{\log 2})$.
In such a case, the value $s/2^m$ converted to a floating-point number
will have an error of at most $\ulp(x)$.

\Paragraph{Evaluation of $R(n)$.}
We estimate $R(n)$ using the terms up to $k=n-2$, again 
as in \cite{Brent78}:
\[ R'(n) = \frac{e^{-n}}{n} \sum_{k=0}^{n-2} \frac{k!}{(-n)^k}. \]
% = \frac{\exp(-n)}{n^{n-1}} \sum_{k=0}^{n-2} (-1)^k \frac{k!} {n^{n-2-k}}.
% Here again, the integer sum is computed exactly, converting it to a 
% floating-point number introduces at most one ulp of error,
% $\exp(-n)$ is computed within one ulp,
% and $n^{n-1}$ within at most $n-2$ ulps.
% The division by $n^{n-1}$ and the multiplication introduce one more ulp of
% error, thus the total error on $R'(n)$ is at most $n+2$ ulps.
Let us introduce $I_k = \int_n^{\infty} \frac{e^{-u}}{u^k} du$.
We have $R(n) = I_1$ and the recurrence $I_k = \frac{e^{-n}}{n^k} - k I_{k+1}$,
which gives
\[ R(n) = \frac{e^{-n}}{n} \sum_{k=0}^{n-2} \frac{k!}{(-n)^k}
        + (-1)^{n-1} (n-1)! I_n. \]
Bounding $n!$ by $(n/e)^n \sqrt{2 \pi (n+1)}$ which holds\footnote{
Formula 6.1.38 from \cite{AbSt73} gives
$x! = \sqrt{2\pi} x^{x+1/2} e^{-x+\frac{\theta}{12x}}$ for $x>0$ and
$0 < \theta < 1$.
Using it for $x \ge 1$, we have $0 < \frac{\theta}{6x} < 1$, and
$e^t < 1+2t$ for $0 < t < 1$, thus 
$(x!)^2 \le 2\pi x^{2x} e^{-2x} (x+\frac{1}{3})$.}
for $n \ge 1$, we have:
\[ |R(n) - R'(n)| = (n-1)! I_n 
        \le \frac{n!}{n} \int_n^{\infty} \frac{e^{-n}}{u^n} du
        \le \frac{n^{n-1}}{e^n} \sqrt{2 \pi (n+1)} \frac{e^{-n}}{(n-1) 
        n^{n-1}} \]
and since $\sqrt{2 \pi (n+1)}/(n-1) \le 1$ for $n \ge 9$:
\[ |R(n) - R'(n)| \le e^{-2n} \quad \mbox{for $n \ge 9$}. \]
Thus we have:
\[ |\gamma - S'(n) - R'(n) - \log n| \le 3 e^{-2n} \quad \mbox{for $n\ge 9$}.\]
% If the working precision is $p$, then choose $n \ge \frac{\log 2}{2} (p+2)$
% such that $3 e^{-2n}$ represents at most one ulp.

To approximate $R'(n)$, we use the following:
\begin{quote}
$m \leftarrow {\rm prec}(x) - \lfloor \frac{n}{\log 2} \rfloor$ \\
$a \leftarrow 2^m$ \\
$s \leftarrow 1$ \\
{\bf for} $k$ {\bf from} $1$ {\bf to} $n$ {\bf do} \\
\q $a \leftarrow \lfloor \frac{k a}{n} \rfloor$ \\
\q $s \leftarrow s + (-1)^{k} a$ \\
$t \leftarrow \lfloor s/n \rfloor$ \\
$x \leftarrow t/2^m$ \\
return $r = e^{-n} x$
\end{quote}
The absolute error $\epsilon_k$ on $a$ at step $k$ satisfies
$\epsilon_k \le 1 + k/n \epsilon_{k-1}$ with $\epsilon_0=0$.
As $k/n \le 1$, we have $\epsilon_k \le k$, whence the error
on $s$ is bounded by $n(n+1)/2$, and that on $t$ by 
$1 + (n+1)/2 \le n+1$ since $n \ge 1$.
The operation $x \leftarrow t/2^m$ is exact as soon as ${\rm prec}(x)$ is large
enough, thus the error on $x$ is at most $(n+1) \frac{e^n}{2^{{\rm prec}(x)}}$.
If $e^{-n}$ is computed with $m$ bits, then
the error on it is at most $e^{-n} 2^{1-m}$.
The error on $r$ is then $(n + 1 + 2/n) 2^{-{\rm prec}(x)} +
\ulp(r)$.
Since $x \ge \frac{2}{3} n$ for $n \ge 2$, and $x 2^{-{\rm prec}(x)}
< \ulp(x)$, this gives an error bounded by 
$\ulp(r) + (n + 1 + 2/n) \frac{3}{2n} \ulp(r)
\le 4 \ulp(r)$ for $n \ge 2$ --- if ${\rm prec}(x) = {\rm prec}(r)$.
Now since $r \le \frac{e^{-n}}{n} \le \frac{1}{8}$, that error
is less than $\ulp(1/2)$.

\Paragraph{Final computation.} We use the formula
$\gamma = S(n) - R(n) - \log n$ with $n$ such that $e^{-2n} \le 
\ulp(1/2) = 2^{-{\rm prec}}$, i.e.~$n \ge {\rm prec} \frac{\log 2}{2}$:
\begin{quote}
$s \leftarrow S'(n)$ \\
$l \leftarrow \log(n)$ \\
$r \leftarrow R'(n)$ \\
return $(s - l) - r$
\end{quote}
Since the final result is in $[\frac{1}{2}, 1]$, and $R'(n) \le 
\frac{e^{-n}}{n}$, then $S'(n)$ approximates $\log n$.
If we use $m + \lceil \log_2({\rm prec}) \rceil$ bits to evaluate $s$ and $l$,
then the error on $s-l$ will be at most $3 \ulp(1/2)$,
so the error on $(s - l) - r$ is at most $5 \ulp(1/2)$,
and adding the $3 e^{-2n}$ truncation error, we get a bound of
$8 \ulp(1/2)$.

\subsubsection{A faster formula}

Brent and McMillan give in \cite{BrMc80} a faster algorithm (B2) using the
modified Bessel functions, which was
used by Gourdon and Demichel to compute 108,000,000 digits of $\gamma$ in
October 1999:
\[ \gamma = \frac{S_0 - K_0}{I_0} - \log n, \]
where $S_0 = \sum_{k=1}^{\infty} \frac{n^{2k}}{(k!)^2} H_k$,
$H_k = 1 + \frac{1}{2} + \cdots + \frac{1}{k}$ being the $k$-th harmonic
number,
$K_0 = \sqrt{\frac{\pi}{4n}} e^{-2n} \sum_{k=0}^{\infty}
        (-1)^k \frac{[(2k)!]^2}{(k!)^3 (64n)^k}$,
and $I_0 = \sum_{k=0}^{\infty} \frac{n^{2k}}{(k!)^2}$.

We have $I_0 \ge \frac{e^{2n}}{\sqrt{4 \pi n}}$ (see \cite{BrMc80} page 306).
From the remark following formula 9.7.2 of \cite{AbSt73},
the remainder in the truncated expansion for $K_0$ up to $k$ does not
exceed the $(k+1)$-th term, whence
$K_0 \le \sqrt{\frac{\pi}{4n}} e^{-2n}$ and
$\frac{K_0}{I_0} \le \pi e^{-4n}$ as in formula (5) of \cite{BrMc80}.
Let $I'_0 = \sum_{k=0}^{K-1} \frac{n^{2k}}{(k!)^2}$ with
$K = \lceil \beta n \rceil$, and $\beta$ is the root of
$\beta (\log \beta - 1) = 3$
($\beta \sim 4.971...$) then
\[ |I_0 - I'_0| \le \frac{\beta}{2 \pi (\beta^2-1)} \frac{e^{-6n}}{n}. \]
Let $K'_0 = \sqrt{\frac{\pi}{4n}} e^{-2n} \sum_{k=0}^{4n-1} (-1)^k
        \frac{[(2k)!]^2}{(k!)^3 (64n)^k}$, then bounding by the next term:
\[ |K_0 - K'_0| \le \frac{(8n+1)}{16 \sqrt{2} n} \frac{e^{-6n}}{n}
        \le \frac{1}{2} \frac{e^{-6n}}{n}. \]
We get from this
\[ \left| \frac{K_0}{I_0} - \frac{K'_0}{I'_0} \right| 
        \le \frac{1}{2 I_0} \frac{e^{-6n}}{n} \le \sqrt{\frac{\pi}{n}}
        e^{-8n}. \]
Let $S'_0 = \sum_{k=1}^{K-1} \frac{n^{2k}}{(k!)^2} H_k$,
then using $\frac{H_{k+1}}{H_k} \le \frac{k+1}{k}$ and the same bound $K$
than for $I'_0$ ($4n \le K \le 5n$), we get:
\[ |S_0 - S'_0| \le \frac{\beta}{2 \pi (\beta^2-1)} H_k \frac{e^{-6n}}{n}. \]
We deduce:
\[ \left| \frac{S_0}{I_0} - \frac{S'_0}{I'_0} \right|
        \le e^{-8n} H_K \frac{\sqrt{4 \pi n}}{\pi (\beta^2-1)}
        \frac{\beta}{n} \le e^{-8n}. \]
Hence we have
\[ \left| \gamma - \left( \frac{S'_0 - K'_0}{I'_0} - \log n \right) \right|
        \le (1 + \sqrt{\frac{\pi}{n}}) e^{-8n} 
        \le 3 e^{-8n}. \]

\section{High level functions}

\subsection{The cosine function}

To evaluate $\cos x$ with a target precision of $n$ bits, we use the following
algorithm with working precision $m$:
\begin{quote}
$k \leftarrow \lfloor \sqrt{n/2} \rfloor$ \\
$r \leftarrow x^2$ rounded up \\ % err <= ulp(r)
$r \leftarrow r/2^{2k}$ \\ % err <= ulp(r)
$s \leftarrow 1, t \leftarrow 1$ \\ % err = 0
{\bf for} $l$ {\bf from} $1$ {\bf while} ${\rm EXP}(t) \ge -m$ \\
\q $t \leftarrow t \cdot r$ rounded up \\ % err <= (3*l-1)*ulp(t)
\q $t \leftarrow \frac{t}{(2l-1)(2l)}$ rounded up \\ % err <= 3*l*ulp(t)
\q $s \leftarrow s + (-1)^l t$ rounded down\\ % err <= l/2^m
{\bf do} $k$ times \\
\q $s \leftarrow 2 s^2$ rounded up \\
\q $s \leftarrow s - 1$ \\
return $s$ \\
\end{quote}
The error on $r$ after $r \leftarrow x^2$
is at most $1 \ulp(r)$ and remains $1 \ulp(r)$ after
$r \leftarrow r/2^{2k}$ since that division is just an exponent shift.
By induction, the error on $t$ after step $l$ of the for-loop is at most
$3 l \ulp(t)$.
Hence as long as $3 l \ulp(t)$ remains less than $\le 2^{-m}$
during that loop
(this is possible as soon as $r < 1/\sqrt{2}$)
and the loop goes to $l_0$, the error on $s$ after the for-loop is at most
$2 l_0 2^{-m}$ (for $|r| < 1$, it is easy to check that $s$ will remain
in the interval $[\frac{1}{2}, 1[$, thus $\ulp(s) = 2^{-m}$).
(An additional $2^{-m}$ term represents the truncation error,
but for $l=1$ the value of $t$ is exact, giving $(2 l_0 - 1) + 1 = 2 l_0$.)

Denoting by $\epsilon_i$ the maximal error on $s$ after the $i$th step
in the do-loop, we have $\epsilon_0 = 2 l_0 2^{-m}$ and
$\epsilon_{k+1} \le 4 \epsilon_k + 2^{-m}$,
giving $\epsilon_k \le (2 l_0+1/3) 2^{2k-m}$.

\subsection{The sine function}

The sine function is computed from the cosine, with a working precision of
$m$ bits:
\begin{quote}
$c \leftarrow \cos x$ rounded to zero \\
$t \leftarrow c^2$ rounded up \\
$u \leftarrow 1 - t$ rounded to nearest \\
$s \leftarrow {\rm sign}(x) \sqrt{u}$ rounded to nearest \\
\end{quote}
The absolute error on $c$ is at most $\ulp(c)
\le 2^{-m}$ since $|c| < 1$,
then that on
$t$ is at most $3 \cdot 2^{-m}$, that on $u$ is at most $3 \cdot 2^{-m} + 
\frac{1}{2}\ulp(u) \le \frac{7}{2} \cdot 2^{-m}$
since $|t|, |u| < 1$, so that on $s$ is at most ${\rm error}(\sqrt{u})
 + \frac{1}{2} \ulp(s) \le 2^{-m} (1/2 + \frac{7}{4 \sqrt{u}})$.
(By Rolle's theorem,
$|\sqrt{u} - \sqrt{u'}| \le \frac{1}{2 \sqrt{v}} |u-u'|$ for
$v \in [u, u']$.)
% |s'-s| <= ulp(s) + |sqrt(u)-sqrt(u')| <= ulp(s) + (u-u')/(2*sqrt(v))
% for v in [u,u']
Let $u = m 2^{-2e}$ with $1 \le m < 4$ and $e \ge 1$, then 
the error on $s$ is at most $2^{-m} (1/2 + \frac{7 \cdot 2^{e}}{4})
        \le 2^{e + 1 - m}$.

\subsection{The tangent function}

The tangent function is computed from the cosine, 
using $\tan x = {\rm sign}(x) \sqrt{\frac{1}{\cos^2 x} - 1}$,
with a working precision of $m$ bits:
\begin{quote}
$c \leftarrow \cos x$ rounded down \\ % c <= cos(x) <= 1
$t \leftarrow c^2$ rounded down \\    % t <= cos(x)^2 <= 1
$v \leftarrow 1/t$ rounded up \\      % v >= 1/cos(x)^2 >= 1
$u \leftarrow v - 1$ rounded up \\    % u >= 1/cos(x)^2 - 1
$s \leftarrow {\rm sign}(x) \sqrt{u}$ rounded away from $0$ \\
\end{quote}
The absolute error on $c$ is at most $\ulp(c)$.
Hence the error on
$t$ is at most $\ulp(t) + 2 c \ulp(c) \le 5 \ulp(t)$,
% err(t) <= ulp(t) + |c^2-c'^2| <= ulp(t) + |c-c'|*|c+c'|
%        <= ulp(t) + 2*ulp(c)*c <= 5*ulp(t)
that on $v$ is at most $\ulp(v) + 5 \ulp(t)/t^2
        \le \ulp(v) + 10 \ulp(1/t) \le 11 \ulp(v)$,
% err(v) <= ulp(v) + |1/t-1/t'| <= ulp(v) + |t-t'|/t/t'
%        <= ulp(v) + 5*ulp(t)/t^2 <= ulp(v) + 10*ulp(1/t) <= 11*ulp(v)
that on $u$ is at most $\ulp(u) + 11 \ulp(v) \le (1 + 11 \cdot 2^e) \ulp(u)$
where $e$ is the exponent difference between $v$ and $u$.
% err(u) <= ulp(u) + err(v) <= ulp(v) + 11*ulp(v) <= (1+11*2^e)*ulp(u)
The final error on $s$ is $\le \ulp(s) + (1+11 \cdot 2^e) \ulp(u)/2/\sqrt{u}
        \le \ulp(s) + (1+11 \cdot 2^e) \ulp(u/\sqrt{u})
        \le (2 + 11 \cdot 2^e) \ulp(s)$.
% err(s) <= ulp(s) + |u-u'|/2/sqrt(u) <= ulp(s) + (1+11*2^e)*ulp(u)/2/sqrt(u)
%        <= ulp(s) + (1+11*2^e)*ulp(u/sqrt(u))
%        <= (2+11*2^e)*ulp(s) 

\subsection{The exponential function}

The {\tt mpfr\_exp} function implements three different algorithms.
For very large precision, it uses a $\O(M(n) \log^2 n)$ algorithm
based on binary splitting, based on the generic implementation for
hypergeometric functions in the file {\tt generic.c} (see \cite{Jeandel00}).
Currently this third algorithm is used only for precision greater
than $13000$ bits.

For smaller precisions, it uses Brent's method~;
if $r = (x - n \log 2)/2^k$ where $0 \le r < \log 2$, then 
\[ \exp(x) = 2^n \cdot \exp(r)^{2^k} \]
and $\exp(r)$ is computed using the Taylor expansion:
\[ \exp(r) =  1 + r + \frac{r^2}{2!} + \frac{r^3}{3!} + \cdots \]
As $r < 2^{-k}$, if the target precision is $n$ bits, then only
about $l = n/k$ terms of the Taylor expansion are needed.
This method thus requires the evaluation of the Taylor series to
order $n/k$, and $k$ squares to compute $\exp(r)^{2^k}$.
If the Taylor series is evaluated using a na\"{\i}ve way, the optimal
value of $k$ is about $n^{1/2}$, giving a complexity of $\O(n^{1/2} M(n))$.
This is what is implemented in {\tt mpfr\_exp2\_aux}.

If we use a baby step/giant step approach, the Taylor series
can be evaluated in $\O(l^{1/2})$ operations, 
thus the evaluation requires $(n/k)^{1/2} + k$ multiplications,
and the optimal $k$ is now about $n^{1/3}$,
giving a total complexity of $\O(n^{1/3} M(n))$.
This is implemented in the function {\tt mpfr\_exp2\_aux2}.

\subsection{The error function}

The error function admits the following expansion at zero:
% \cite[formula 7.1.5]{AbSt73}:
% \[ {\rm erf} \, z = \frac{2}{\sqrt{\pi}} \sum_{k=0}^{\infty} \frac{(-1)^k}
%       {k! (2k+1)} z^{2k+1}, \]
\cite[formula 7.1.6]{AbSt73}:
\[ {\rm erf} \, z = \frac{2}{\sqrt{\pi}} e^{-z^2}
        \sum_{k=0}^{\infty} \frac{2^k}{1 \cdot 3 \cdots (2k+1)} z^{2k+1}, \]
and the following asymptotic expansion for ${\rm erfc} z = 1 - {\rm erf} z$
\cite[formula 7.1.23]{AbSt73}:
\[ \sqrt{\pi} z e^{z^2} {\rm erfc} \, z \sim 1 +
        \sum_{k=1}^{\infty} (-1)^k \frac{1 \cdot 3 \cdots (2k-1)}{(2z^2)^k}. \]
The former formula requires $m \sim n \frac{\log 2}{\log(m/(ez^2))}$ terms
% same number of terms for 7.1.5 and 7.1.6
to get $n$ correct bits, and the latter requires
$m \sim n \frac{\log 2}{\log(ez^2/m)}$ terms.
Thus, we use the expansion at $z=0$ for $n \ge e z^2$, and the asymptotic
expansion for $n < e z^2$.

\medskip

If we use the series at $z=0$, the maximum term is obtained for
$k \sim z^2$, and is of the order of $e^{z^2}$; this means
we need $z^2/(\log 2)$ additional bits. As $z^2 \le n/e$ in that case,
this is bounded by $n/(e \log 2) \sim 0.531 n$.


The series at $z=0$ is implemented as follows, 
$m$ representing the working precision,
$x, y, s, t, u$ being integer variables, and $p, r$ floating-point
variables:
\begin{quote}
\verb|erf_0|$(z, n)$, assumes $z^2 \le n/e$ \\
$m \leftarrow n + z^2/(\log 2)$ \\
$x \leftarrow \lceil {\rm msb}(z, m) \rceil$ \\
$y \leftarrow \lceil {\rm msb}(x^2,m) \rceil$ such that $y \sim z^2 2^{e_y}$ \\
$s \leftarrow 2^n, t \leftarrow 2^n$ \\
{\bf for} $k$ {\bf from} $1$ {\bf do} \\
\q $t \leftarrow \lceil y t/k \rceil$ \\
\q $t \leftarrow \lceil t/2^{e_y} \rceil$ \\
\q $u \leftarrow \lceil t/(2k+1) \rceil$ \\
\q $s \leftarrow {\mathcal N}(s + (-1)^k u)$ \\
\q {\bf if} $u \le 1$ {\bf then} break \\
$r \leftarrow 2 \lceil z s/2^n \rceil$ \\
$p \leftarrow \pi, p \leftarrow \sqrt{p}$ \\
return $r/p$
\end{quote}
The variable $u$ contains the current term $\frac{z^{2k}}{k! (2k+1)}$,
scaled by $2^m$. Suppose $u \le 1$ for index $k_0$:
as $u \ge 2$ for index $k_0-1$, and the 
ratio between two consecutive terms decreases, then $u \le 1/2$ for index 
$k_0+1$ and the alternating series $\sum_{k_0+1}^{\infty} \frac{(-1)^k
z^{2j}}{k! (2k+1)}$
is bounded by its first term, i.e.~$2^{-m-1}$ after rescaling.

Now the relative error on $x$ is at most $2^{1-n}$,
that on $y$ is at most $2x/2^m + 1$,
that on $s$ and $t$ is zero initially.
Let $\varepsilon_k$ and $\tau_k$ the errors on $y$ and $t$
at the beginning of loop $k$,
then that for $t$ after $t \leftarrow \lceil y t/2^m \rceil$
is at most $(\varepsilon_k t + \tau_k y)/2^m + 1$.

\subsection{Generic error of addition/soustraction}\label{generic:sous}


We want to compute the generic error of the soustraction, this following rules can be apply on addition too.

\begin{eqnarray}\nonumber
\textnormal{Note:}&& error(u) \leq k_u \, \ulp(u), \;\; error(v) \leq k_v \, \ulp(v)
\end{eqnarray}

\begin{eqnarray}\nonumber
\textnormal{Note:}&& \ulp(w)=2^{e_w-p}, \;\; \ulp(u)=2^{e_u-p},\;\; \ulp(v)=2^{e_v-p}\;\;\textnormal{with} \; p \; \textnormal{the accuracy} \\\nonumber
&& \ulp(u)=2^{d+e_w-p}, \;\; \ulp(u)=2^{d+e_w-p},\;\;\textnormal{with} \;\;d=e_u-e_w \;\; d^{'}=e_v-e_w 
\end{eqnarray}
\begin{eqnarray}\nonumber
error(w)& \leq &c_w \ulp(w) + k_u \ulp(u) + k_v \ulp(v) \\\nonumber
&\leq&(c_w+k_u 2^d+ k_v 2^{d^{'}}) \ulp(w)
\end{eqnarray}
\begin{eqnarray}\nonumber
&&\textnormal{If} \;\; ( u \geq 0  \;\;\textnormal{and}\;\;  v \geq 0) \;\;\textnormal{or}\;\; (u \leq 0 \;\;\textnormal{and}\;\; v \leq 0)
\end{eqnarray}
\begin{eqnarray}\nonumber
error(w)& \leq&(c_w + k_u + k_v) \, \ulp(w)
\end{eqnarray}
\begin{eqnarray}\nonumber
\textnormal{Note:}&&\textnormal{If}\;\; w=N(u+v) \;\;\textnormal{Then}\;\; c_w =\frac{1}{2} \;\;\textnormal{else}\;\; c_w =1\\\nonumber
\end{eqnarray}

\subsection{Generic error of multiplication}\label{generic:mul}


We want to compute the generic error of the multiplication.

\begin{eqnarray}\nonumber
w&=&o(u.v) \\\nonumber
\textnormal{Note:}&& error(u) \leq k_u \, \ulp(u), \;\; error(v) \leq k_v \, \ulp(v)
\end{eqnarray}
\begin{eqnarray}\nonumber
error(w)& = &|w - x.y| \\\nonumber
& \leq &|w - u.b| +|u.y - x.y| \\\nonumber
& \leq & c_w \ulp(w) +  \frac{1}{2} [|u.v-u.y|+|u.y-x.y|+|u.v-x.v|+|x.v-x.y|]\\\nonumber
& \leq & c_w \ulp(w) +  \frac{u+x}{2} k_v \ulp(v) + \frac{v+y}{2} k_u \ulp(u)\\\nonumber
& \leq & c_w \ulp(w) +  \frac{u(1+c_u^+)}{2} k_v \ulp(v) + \frac{v(1+c_v^+)}{2} k_u \ulp(u) \U{R9}\\\nonumber
& \leq & c_w \ulp(w) +  (1+c_u^+) k_v \ulp(u.v) + (1+c_v^+) k_u \ulp(u.v) \U{R4}\\\nonumber
& \leq & [ c_w +  (1+c_u^+) k_v + (1+c_v^+) k_u ] \ulp(w)\U{R8}\\\nonumber
\end{eqnarray}
\begin{eqnarray}\nonumber
\textnormal{Note:}&&\textnormal{If}\;\; w=N(u+v) \;\;\textnormal{Then}\;\; c_w =\frac{1}{2} \;\;\textnormal{else}\;\; c_w =1
\end{eqnarray}

\subsection{Generic error of inverse}\label{generic:inv}

We want to compute the generic error of the inverse.

\begin{eqnarray}\nonumber
w&=&o(\frac{1}{v}) \\\nonumber
\textnormal{Note:}&& error(u) \leq k_u \, \ulp(u)
\end{eqnarray}
\begin{eqnarray}\nonumber
error(w)& = &|w - \frac{1}{x}| \\\nonumber
& \leq &|w - \frac{1}{u}| +|\frac{1}{u} - \frac{1}{x}| \\\nonumber
& \leq & c_w \ulp(w) + \frac{1}{ux}|u-x| \\\nonumber
& \leq & c_w \ulp(w) + \frac{k_u}{ux} \ulp(u)
\end{eqnarray}
\begin{eqnarray}\nonumber
\textnormal{Note:}&& \frac{u}{c_u} \leq x\;\; \U{R6}\\\nonumber
&&\n{for } u=\minf(x),\;c_u=1 \n{ else } c_u=2\\\nonumber
&& \n{then: } \frac{1}{x} \leq c_u \frac{1}{u} 
\end{eqnarray}
\begin{eqnarray}\nonumber
error(w)& \leq & c_w \ulp(w) + c_u\frac{k_u}{u^2} \ulp(u)\\\nonumber
& \leq & c_w \ulp(w) + 2.c_u.k_u \ulp(\frac{u}{u^2}) \U{R4}\\\nonumber
& \leq & [c_w + 2.c_u.k_u].\ulp(w) \U{R8}
\end{eqnarray}
\begin{eqnarray}\nonumber
\textnormal{Note:}&&\textnormal{If}\;\; w=N(\frac{1}{u}) \;\;\textnormal{Then}\;\; c_w =\frac{1}{2} \;\;\textnormal{else}\;\; c_w =1\\\nonumber\end{eqnarray}

\subsection{Generic error of division}\label{generic:div}


We want to compute the generic error of the division.

\begin{eqnarray}\nonumber
w&=&o(\frac{u}{v}) \\\nonumber
\textnormal{Note:}&& error(u) \leq k_u \, \ulp(u), \;\; error(v) \leq k_v \, \ulp(v)
\end{eqnarray}
\begin{eqnarray}\nonumber
error(w)& = &|w - \frac{x}{y}| \\\nonumber
& \leq &|w - \frac{u}{v}| +|\frac{u}{v} - \frac{x}{y}| \\\nonumber
& \leq & c_w \ulp(w) + \frac{1}{vy}|uy-vx| \\\nonumber
& \leq & c_w \ulp(w) + \frac{1}{vy}[|uy-xy|+|xy-vx| ]\\\nonumber
& \leq & c_w \ulp(w) + \frac{1}{vy}[y k_u \ulp(u)+ x k_v \ulp(v)]\\\nonumber
& \leq & c_w \ulp(w) + \frac{k_u}{v}  \ulp(u)+ \frac{k_v x}{vy} \ulp(v)
\end{eqnarray}
\begin{eqnarray}\nonumber
\textnormal{Note:}&& \frac{\ulp(u)}{v} \leq  2 \ulp(\frac{u}{v}) \;\; \U{R4}\\\nonumber
&& 2 \ulp(\frac{u}{v}) \leq  2 \ulp(w) \;\; \U{R8}
\end{eqnarray}
\begin{eqnarray}\nonumber
\textnormal{Note:}&& x \leq c_u u \textnormal{ and } \frac{v}{c_v} \leq y\;\; \U{R6}\\\nonumber
&&\n{with } \n{for } u=\pinf(x),\;c_u=1 \n{ else } c_u=2\\\nonumber
&&\n{ and }\n{for } v=\minf(y),\;c_v=1 \n{ else } c_v=2\\\nonumber
&& \n{then: } \frac{x}{y} \leq c_u c_v  \frac{u}{v} 
\end{eqnarray}
\begin{eqnarray}\nonumber
error(w)& \leq & c_w \ulp(w) + 2.k_u  \ulp(w)+ c_u.c_v.\frac{k_v u}{vv} \ulp(v)\\\nonumber
& \leq & c_w \ulp(w) + 2.k_u  \ulp(w)+ 2.c_u.c_v.k_v \ulp(\frac{u.v}{v.v}) \U{R4}\\\nonumber
& \leq & [c_w  + 2.k_u+ 2.c_u.c_v.k_v].\ulp(w) \U{R8}
\end{eqnarray}
\begin{eqnarray}\nonumber
\textnormal{Note:}&&\textnormal{If}\;\; w=N(\frac{u}{v}) \;\;\textnormal{Then}\;\; c_w =\frac{1}{2} \;\;\textnormal{else}\;\; c_w =1
\end{eqnarray}

\subsection{Generic error of square root}\label{generic:sqrt}


We want to compute the generic error of the square root.

\begin{eqnarray}\nonumber
v&=&o(\sqrt{u}) \\\nonumber
\textnormal{Note:}&& error(u) \leq k_u \, \ulp(u)\\\nonumber
\end{eqnarray}
\begin{eqnarray}\nonumber
error(v)& = &|v - \sqrt{x}| \\\nonumber
& \leq &|v - \sqrt{u}| +|\sqrt{u} - \sqrt{x}| \\\nonumber
& \leq & c_v \ulp(v) + \frac{1}{\sqrt{u} + \sqrt{x}}|u-x| \\\nonumber
& \leq & c_v \ulp(v) + \frac{1}{\sqrt{u} + \sqrt{x}} k_u \ulp(u) \\\nonumber
\end{eqnarray}
\begin{eqnarray}\nonumber
\textnormal{Note:}&& u.c_u^- \leq x \;\; \U{R9}\\\nonumber
 && \sqrt{u.c_u^-} \leq \sqrt{x} \\\nonumber 
 && \sqrt{u}.(1+\sqrt{c_u^-}) \leq \sqrt{x}+\sqrt{u} \\\nonumber 
 && \frac{1}{\sqrt{x}+\sqrt{u}} \leq
    \frac{1}{\sqrt{u}.(1+\sqrt{c_u^-})} 
\end{eqnarray}
\begin{eqnarray}\nonumber
error(v)& \leq & c_v \ulp(v) + 
                 \frac{1}{\sqrt{u}.(1+\sqrt{c_u^-})}  k_u \ulp(u) \\\nonumber 
& \leq & c_v \ulp(v) + \frac{2}{1+\sqrt{c_u^-}}  
       k_u \ulp(\sqrt{u}) \;\; \U{R4}\\\nonumber 
& \leq & (c_v + \frac{2.k_u}{1+\sqrt{c_u^-}})  \ulp(v) \;\; \U{R8}\\\nonumber 
\end{eqnarray}

\subsection{Generic error of the exponential }\label{generic:exp}


We want to compute the generic error of the exponential.

\begin{eqnarray}\nonumber
v&=&o(e^{u}) \\\nonumber
\textnormal{Note:}&& error(u) \leq k_u \, \ulp(u)\\\nonumber
\end{eqnarray}
\begin{eqnarray}\nonumber
error(v)& = &|v - e^{x}| \\\nonumber
& \leq &|v - e^{u}| +|e^{u} - e^{x}| \\\nonumber
& \leq & c_v \ulp(v) +  e^t |u-x| \n{ with Rolle's theorem, for } t\in[x,u]\n{ or }t\in[u,x]
\end{eqnarray}
\begin{eqnarray}\nonumber
error(v)& \leq & c_v \ulp(v) +  c_u^* e^u k_u \ulp(u) \\\nonumber
& \leq & c_v \ulp(v) +  2 c_u^* u k_u \ulp(e^u) \;\U{R4}\\\nonumber
& \leq & (c_v +  2 c_u^* u k_u )\ulp(v) \;\U{R8}\\\nonumber
& \leq & (c_v +  c_u^* 2^{\Exp(u)+1} k_u )\ulp(v)
\end{eqnarray}
\begin{eqnarray}\nonumber
\textnormal{Note:}&& u= m_u 2^{e_u} \n{ and }\ulp(u)=2^{e_u-p} \n{ with } p \n{ the accuracy} \\\nonumber
\n{ Case }&x \leq u&  c_u^* =1 \\\nonumber
\n{ Case }&u \leq x& \\\nonumber
&&  x \leq u + k_u \ulp(u)\\\nonumber
&& e^x \leq e^u e^{k_u \ulp(u)}\\\nonumber
&&e^x \leq e^u e^{k_u 2^{e_u-p}}\\\nonumber
&\n{then}& c_u^* = e^{k_u 2^{\Exp(u)-p}}\\\nonumber
\end{eqnarray}



\subsection{Generic error of the logarithm }\label{generic:log}


We want to compute the generic error of the logarithm.

\begin{eqnarray}\nonumber
v&=&o(\log{u}) \\\nonumber
\textnormal{Note:}&& error(u) \leq k_u \, \ulp(u)\\\nonumber
\end{eqnarray}
\begin{eqnarray}\nonumber
error(v)& = &|v - \log{x}| \\\nonumber
& \leq &|v - \log{u}| +|\log{u} - \log{x}| \\\nonumber
& \leq & c_v \ulp(v) + \log{|\frac{x}{u}|} \\\nonumber
& \leq & c_v \ulp(v) + \frac{|x-u|}{u} \\\nonumber
& \leq & c_v \ulp(v) + \frac{k_u \, \ulp(u)}{u}\\\nonumber
& \leq & c_v \ulp(v) + k_u \, \ulp(1)\;\; \U{R7}\\\nonumber
\end{eqnarray}
\begin{eqnarray}\nonumber
\textnormal{Note:}&& \ulp(1)=2^{1-p}, \n{ and } \ulp(v)=2^{e_v-p} \n{ with } p \n{ the accuracy} \\\nonumber
&& \ulp(1)= 2^{1-e_v+e_v-p}=2^{1-e_v} \ulp(v)
\end{eqnarray}
\begin{eqnarray}\nonumber
error(v)& \leq & c_v \ulp(v) + k_u 2^{1-e_v} \ulp(v)\\\nonumber
& \leq & (c_v + k_u 2^{1-e_v} )\ulp(v)\\\nonumber
\end{eqnarray}

\subsection{The hyperbolic cosine function}

The {\tt mpfr\_cosh} ($\cosh{x}$) function implements the hyperbolic
cosine as :

$$
\cosh x = \frac{1}{2} \left( e^{x} + \frac{1}{e^x} \right) 
$$

The algorithm used for the calculation of the hyperbolic cosine is as follows\footnote{$o()$ represent the arrondi error and $error(u)$ the
  error associate with the calcualtion of $u$}:

\begin{eqnarray}\nonumber
u&\leftarrow&o(e^x)\\\label{coshalgo1}
v&\leftarrow&o({u}^{-1})\\\label{coshalgo2}
w&\leftarrow&o(u+v)\\\label{coshalgo3}
s&\leftarrow&\frac{1}{2} w\\\label{coshalgo4}
\end{eqnarray}

Now, we have to bound the rounding error for each step of this
algorithm.  First, let consider the parity of hyperbolic cosine
($\cosh(-x)=\cosh(x)$) : the problem is reduced to calculate $\cosh x$
with $x \geq 0$. We can deduce $e^x \geq 1$ and $0 \leq e^{-x} \leq
1$.



\begin{center}
\begin{tabular}{l l l}

\begin{minipage}{2.5cm}


${\textnormal{error}}(u)$


$u \leftarrow o(e^x)$\\
$-\infty \;\; (\bullet)$

\end{minipage} &
\begin{minipage}{7.5cm}

\begin{eqnarray}\nonumber
  |u-e^x| &\leq& \ulp(u)\\\nonumber
\end{eqnarray}

\end{minipage} &
\begin{minipage}{6cm}
{\hspace{7cm}}
\end{minipage}\\%\hline
\begin{minipage}{2.5cm}
${\textnormal{error}}(v)$


$v \leftarrow o({u}^{-1}) $\\
$+\infty \;\; (\bullet\bullet)$ 
\end{minipage} &
\begin{minipage}{7.5cm}



\begin{eqnarray}\nonumber
  &&|v-e^{-x}| \\\nonumber
  &       \leq&  |v - u^{-1}| +  |u^{-1}  - e^{-x}|\\\nonumber
  &       \leq& \ulp(v) + \frac{1}{u \cdot e^x} |u-e^{x}|\\\nonumber
  &       \leq& \ulp(v) + \frac{1}{u^2} \ulp(u)\;\;(\star)\\\nonumber
  &       \leq& \ulp(v) + 2 \ulp(\frac{1}{u})\;\;(\star\star)\\\nonumber
  &       \leq& 3 \, \ulp(v)\;\;(\star\star\star)
\end{eqnarray}


\end{minipage} &
\begin{minipage}{6cm}

$(\star)$

With $\frac{1}{e^x} \leq \frac{1}{u}$,\\ 
for that we must have $u \leq e^x$,\\
it is possible with a rounding of\\
$u$ to $-\infty \;\; (\bullet)$

$(\star\star)$

From inequation \U{R4}, 
\[   a \cdot \ulp(b) \leq 2 \cdot \ulp(a \cdot b)\]
if $a =\frac{1}{u^2},\;b = u$ then  
\[ \frac{1}{u^2} \ulp(u)  \leq 2 \ulp(\frac{1}{u})\]

$(\star\star\star)$

If $\ulp(\frac{1}{u}) \leq ulp(v)$,\\
it is possible with a rounding of \\
$v$ to $+\infty \;\; (\bullet)$\\



\end{minipage}\\%\hline
\begin{minipage}{2.5cm}
${\textnormal{error}}(w)$


$w \leftarrow o(u+v) $
\end{minipage} &
\begin{minipage}{7.5cm}



\begin{eqnarray}\nonumber
  &&|w-(e^{x}+e^{-x})| \\\nonumber
  &       \leq&  |w - (u+v)|+|u - e^x|+|v - e^{-x}|\\\nonumber
  &       \leq& \ulp(w) + \ulp(u) + 3\ulp(v)\\\nonumber
  &       \leq& \ulp(w) + 4\ulp(u)\;\;(\star)\\\nonumber
  &       \leq& 5\ulp(w)\;\;(\star\star)\\\nonumber
\end{eqnarray}


\end{minipage} &
\begin{minipage}{6cm}

$(\star)$

With $v \leq 1\leq u$ 

then $\ulp(v) \leq \ulp(u)$

$(\star\star)$

With $u \leq w$ 

then $\ulp(u) \leq \ulp(w)$

\end{minipage}\\%\hline
\begin{minipage}{2.5cm}
${\textnormal{error}}(s)$

$s \leftarrow o(\frac{w}{2}) $
\end{minipage} &
\begin{minipage}{7.5cm}

\begin{center}


\begin{eqnarray}\nonumber
 {\textnormal{error}}(s) & = &  {\textnormal{error}}(w)\\\nonumber
 & \leq &  5\ulp(s)
\end{eqnarray}



\end{center}

\end{minipage} &
\begin{minipage}{6cm}


\end{minipage}


\end{tabular}
\end{center}

That shows the rounding error on the calculation of $\cosh x$ can be
bound by $5 \;\; \ulp$ on the result. So, to calculate the size of
intermediary variables, we have to add, at least, $\lceil \log_2 5 \rceil=3$ bits the wanted
precision.
 
\subsection{The inverse hyperbolic cosine function}

The {\tt mpfr\_acosh} ($\n{acosh}{x}$) function implements the inverse hyperbolic
cosine as :

$$
\n{acosh} = \log \left( \sqrt{x+1} \sqrt{x-1} + x \right) 
$$

The algorithm used for the calculation of the inverse hyperbolic cosine is as follows

\begin{eqnarray}\nonumber
q&\leftarrow&o(x+1)\\\nonumber
r&\leftarrow&o(x-1)\\\nonumber
s&\leftarrow&o(\sqrt{q})\\\nonumber
t&\leftarrow&o(\sqrt{r})\\\nonumber
u&\leftarrow&o(s \times t)\\\nonumber
v&\leftarrow&o(u+x)\\\nonumber
w&\leftarrow&o(\log  v)
\end{eqnarray}

Now, we have to bound the rounding error for each step of this
algorithm. First, let consider the function field : {\tt mpfr\_acosh} is define for $x \geq 1$.

\begin{center}
\begin{tabular}{l l l}

\begin{minipage}{2.5cm}
${\textnormal{error}}(q)$


$q \leftarrow \minf(x+1) $
$(\bullet)$
\end{minipage} &
\begin{minipage}{7.5cm}



\begin{eqnarray}\nonumber
  &&|q-(x+1)| \\\nonumber
  &       \leq&  2 \ulp(q)\;\;(\star)\\\nonumber
\end{eqnarray}


\end{minipage} &
\begin{minipage}{6cm}

($\star$)

see subsection \ref{generic:sous}


\end{minipage}\\%\hline
\begin{minipage}{2.5cm}
${\textnormal{error}}(r)$


$r \leftarrow \minf(x-1) $
$(\bullet\bullet)$
\end{minipage} &
\begin{minipage}{7.5cm}



\begin{eqnarray}\nonumber
  &&|r-(x-1)| \\\nonumber
  &       \leq&  (1+2^{\Exp(x)-\Exp(r)}) \ulp(r)\;\;(\star)\\\nonumber
\end{eqnarray}


\end{minipage} &
\begin{minipage}{6cm}

($\star$)

see subsection \ref{generic:sous}


\end{minipage}\\%\hline
\begin{minipage}{2.5cm}
${\textnormal{error}}(s)$


$s \leftarrow \pinf(\sqrt{q}) $
($\bullet\bullet\bullet$)

\end{minipage} &
\begin{minipage}{7.5cm}

\begin{eqnarray}\nonumber
  &&|s-\sqrt{x+1}| \\\nonumber
  &       \leq& \ulp(s) + |\sqrt{q}-\sqrt{x+1}|\\\nonumber
  &       \leq& \ulp(s) + \frac{1}{\sqrt{q} + \sqrt{x+1}}|q-(x+1)|\\\nonumber
  &       \leq& \ulp(s) + \frac{1}{\sqrt{q} + \sqrt{x+1}}.2.\ulp(q)  \\\nonumber
  &       \leq& \ulp(s) + \frac{1}{2\sqrt{q}}.2.\ulp(q) \;\;(\star) \\\nonumber
  &       \leq& \ulp(s) + 2.\ulp(\sqrt{q}) \;\;(\star\star)\\\nonumber
  &       \leq& (1 + 2) \ulp(s) \;\;(\star\star\star)
\end{eqnarray}


\end{minipage} &
\begin{minipage}{6cm}

($\star$)

If $q \leftarrow \minf(x+1) \;(\bullet)$

Then $q \leq x+1$

or

$\frac{1}{\sqrt{x+1}+\sqrt{v}} \leq \frac{1}{2.\sqrt{q}}$

($\star\star$)

$\U{R4}$

($\star\star\star$)

$\U{R8}$

\end{minipage}\\%\hline
\begin{minipage}{2.5cm}
${\textnormal{error}}(t)$


$t \leftarrow \pinf(\sqrt{r}) $
($\bullet\bullet\bullet\bullet$)

\end{minipage} &
\begin{minipage}{7.5cm}

\begin{eqnarray}\nonumber
  &&|t-\sqrt{x-1}| \\\nonumber
  &       \leq& \ulp(t) + |\sqrt{r}-\sqrt{x-1}|\\\nonumber
  &       \leq& \ulp(t) + \frac{1}{\sqrt{r} + \sqrt{x-1}}|r-(x-1)|\\\nonumber
  &       \leq& \ulp(t) + \frac{1}{\sqrt{r} + \sqrt{x+1}} \\\nonumber
  &      \cdots & (1+2^{\Exp(x)-\Exp(r)}) \ulp(r)  \\\nonumber
  &       \leq& \ulp(t) + \frac{1}{2\sqrt{r}} \cdots \;\;(\star) \\\nonumber
  &\cdots& (1+2^{\Exp(x)-\Exp(r)}) \ulp(r) \;\;(\star) \\\nonumber
  &       \leq& \ulp(t) + (1+2^{\Exp(x)-\Exp(r)})\ulp(\sqrt{r}) \;\;(\star\star)\\\nonumber
  &       \leq& (2+2^{\Exp(x)-\Exp(r)}) \ulp(t) \;\;(\star\star\star)
\end{eqnarray}


\end{minipage} &
\begin{minipage}{6cm}

($\star$)

If $r \leftarrow \minf(x-1) \;(\bullet\bullet)$ 

Then $q \leq x+1$

or

$\frac{1}{\sqrt{x-1}+\sqrt{r}} \leq \frac{1}{2.\sqrt{r}}$

($\star\star$)

$\U{R4}$

($\star\star\star$)

$\U{R8}$

\end{minipage}\\%\hline
\begin{minipage}{2.5cm}
${\textnormal{error}}(u)$


$u \leftarrow o(t \times s) $
\end{minipage} &
\begin{minipage}{7.5cm}

\begin{eqnarray}\nonumber
  &&|u-(\sqrt{x+1} \sqrt{x-1})| \\\nonumber
  &       \leq& (1+2 \times 3 +2 \times (2+2^{\Exp{x}-\Exp{r}}))\\\nonumber
  &       \cdots&  \ulp(w) \;(\star)\\\nonumber
  &       \leq& (13+2^{\Exp{x}-\Exp{r}+1}) \ulp(u)
\end{eqnarray}


\end{minipage} &
\begin{minipage}{6cm}

($\star$)

see subsection \ref{generic:mul}

with $(\bullet\bullet\bullet)$ and $(\bullet\bullet\bullet\bullet)$
\end{minipage}\\%\hline
\begin{minipage}{2.5cm}
${\textnormal{error}}(v)$


$v \leftarrow o(u+x) $
\end{minipage} &
\begin{minipage}{7.5cm}

\begin{eqnarray}\nonumber
  &&|v-(\sqrt{x+1} \sqrt{x-1} +x)| \\\nonumber
  &       \leq& (15+2^{\Exp(x)-\Exp(r)+1}) \ulp(v)
\end{eqnarray}


\end{minipage} &
\begin{minipage}{6cm}

($\star$)

see subsection \ref{generic:sous}

\end{minipage}\\%\hline
\begin{minipage}{2.5cm}
${\textnormal{error}}(w)$


$w \leftarrow o(\log{v}) $
\end{minipage} &
\begin{minipage}{7.5cm}

\begin{eqnarray}\nonumber
  &&|w-\log(\sqrt{x+1} \sqrt{x-1} +x)| \\\nonumber
  &       \leq& (1+(15+2^{\Exp(x)-\Exp(r)+1}).2^{1-\Exp(w)} \ulp(w) \\\nonumber
  &       \leq& (1+15.2^{1-\Exp(w)}+2^{\Exp(x)-\Exp(r)-\Exp(w)+2}) \ulp(w)
\end{eqnarray}


\end{minipage} &
\begin{minipage}{6cm}

($\star$)

see subsection \ref{generic:log}

\end{minipage}
\end{tabular}
\end{center}

That shows the rounding error on the calculation of $\n{acosh} x$ can
be bound by $ (1+15.2^{1-\Exp(w)}+2^{\Exp(x)-\Exp(r)-\Exp(w)+2})\;\;
\ulp$ on the result. So, to calculate the size of intermediary
variables, we have to add, at least, $\lceil \log_2
(1+15.2^{1-\Exp(w)}+2^{\Exp(x)-\Exp(r)-\Exp(w)+2}) \rceil$ bits the
wanted precision.
 
\subsection{The hyperbolic sine function}

The {\tt mpfr\_sinh} ($\sinh{x}$) function implements the hyperbolic
sine as :

$$
\sinh x = \frac{1}{2} \left( e^{x} - \frac{1}{e^x} \right) 
$$

The algorithm used for the calculation of the hyperbolic sine is as follows\footnote{$o()$ represent the arrondi error and $error(u)$ the
  error associate with the calcualtion of $u$}:

\begin{eqnarray}\nonumber
u&\leftarrow&o(e^x)\\\nonumber
v&\leftarrow&o({u}^{-1})\\\nonumber
w&\leftarrow&o(u-v)\\\nonumber
s&\leftarrow&\frac{1}{2} w
\end{eqnarray}

Now, we have to bound the rounding error for each step of this
algorithm.  First, let consider the parity of hyperbolic sine
($\sinh(-x)=-\sinh(x)$) : the problem is reduced to calculate $\sinh x$
with $x \geq 0$. We can deduce $e^x \geq 1$ and $0 \leq e^{-x} \leq
1$.



\begin{center}
\begin{tabular}{l l l}

\begin{minipage}{2.5cm}


${\textnormal{error}}(u)$


$u \leftarrow \minf(e^x)$\\
$(\bullet)$

\end{minipage} &
\begin{minipage}{7.5cm}

\begin{eqnarray}\nonumber
  |u-e^x| &\leq& \ulp(u)\\\nonumber
\end{eqnarray}

\end{minipage} &
\begin{minipage}{6cm}
{\hspace{7cm}}
\end{minipage}\\%\hline
\begin{minipage}{2.5cm}
${\textnormal{error}}(v)$


$v \leftarrow \pinf({u}^{-1}) $\\
$(\bullet\bullet)$ 
\end{minipage} &
\begin{minipage}{7.5cm}



\begin{eqnarray}\nonumber
  &&|v-e^{-x}| \\\nonumber
  &       \leq&  |v - u^{-1}| +  |u^{-1}  - e^{-x}|\\\nonumber
  &       \leq& \ulp(v) + \frac{1}{u \cdot e^x} |u-e^{x}|\\\nonumber
  &       \leq& \ulp(v) + \frac{1}{u^2} \ulp(u)\;\;(\star)\\\nonumber
  &       \leq& \ulp(v) + 2 \ulp(\frac{1}{u})\;\;(\star\star)\\\nonumber
  &       \leq& 3 \, \ulp(v)\;\;(\star\star\star)
\end{eqnarray}


\end{minipage} &
\begin{minipage}{6cm}

$(\star)$

With $\frac{1}{u} \leq \frac{1}{e^x}$,\\ 
for that we must have $e^x \leq u$,\\
it is possible with $u=\minf(e^x)$ $(\bullet)$

$(\star\star)$

From inequation \U{R4}, 
\[   a \cdot \ulp(b) \leq 2 \cdot \ulp(a \cdot b)\]
if $a =\frac{1}{u^2},\;b = u$ then  
\[ \frac{1}{u^2} \ulp(u)  \leq 2 \ulp(\frac{1}{u})\]

$(\star\star\star)$

If $\ulp(\frac{1}{u}) \leq \ulp(v)$,\\
it is possible with $v=\pinf(u^{-1})$ $(\bullet\bullet)$



\end{minipage}\\%\hline
\begin{minipage}{2.5cm}
${\textnormal{error}}(w)$


$w \leftarrow o(u+v) $
\end{minipage} &
\begin{minipage}{7.8cm}



\begin{eqnarray}\nonumber
  &&|w-(e^{x}-e^{-x})| \\\nonumber
  &       \leq&  |w - (u-v)|+|u - e^x|+|-v + e^{-x}|\\\nonumber
  &       \leq& \ulp(w) + \ulp(u) + 3\ulp(v)\\\nonumber
  &       \leq& \ulp(w) + 4\ulp(u)\;\;(\star)\\\nonumber
  &       \leq& (1+ 4 \cdot 2^{\Exp(u)-\Exp(w)}) \ulp(w)\;\;(\star\star)\\\nonumber
\end{eqnarray}


\end{minipage} &
\begin{minipage}{6cm}

$(\star)$

With $v \leq 1\leq u$ 

then $\ulp(v) \leq \ulp(u)$

$(\star\star)$

see subsection \ref{generic:sous}

\end{minipage}\\%\hline
\begin{minipage}{2.5cm}
${\textnormal{error}}(s)$

$s \leftarrow o(\frac{w}{2}) $
\end{minipage} &
\begin{minipage}{7.5cm}

\begin{center}


\begin{eqnarray}\nonumber
 {\textnormal{error}}(s) & = &  {\textnormal{error}}(w)\\\nonumber
 & \leq &  (1+ 4 \cdot 2^{\Exp(u)-\Exp(w)}) \ulp(w)
\end{eqnarray}



\end{center}

\end{minipage} &
\begin{minipage}{6cm}


\end{minipage}


\end{tabular}
\end{center}


That show the rounding error on the calculation of $\sinh x$ can be bound by $(1+ 4 \cdot 2^{\Exp(u)-\Exp(w)}) \ulp(w)$, then the number of bits need to add to the want accuracy to define intermediary variable is :

\[
N_t=\lceil \log_2(1+ 4 \cdot 2^{\Exp(u)-\Exp(w)}) \rceil
\]


\subsection{The inverse hyperbolic sine function}

The {\tt mpfr\_asinh} ($\n{acosh}{x}$) function implements the inverse hyperbolic sine as :

$$
\n{asinh} = \log \left( \sqrt{x^2+1} + x \right) 
$$

The algorithm used for the calculation of the inverse hyperbolic sine is as follows

\begin{eqnarray}\nonumber
s&\leftarrow&o(x^2)\\\nonumber
t&\leftarrow&o(s+1)\\\nonumber
u&\leftarrow&o(\sqrt{t})\\\nonumber
v&\leftarrow&o(u+x)\\\nonumber
w&\leftarrow&o(\log  v)
\end{eqnarray}


Now, we have to bound the rounding error for each step of this
algorithm.  First, let consider the parity of hyperbolic arcsine
($\n{asinh}(-x)=-\n{asinh}(x)$) : the problem is reduced to calculate $\n{asinh} x$
with $x \geq 0$.

\begin{center}
\begin{tabular}{l l l}

\begin{minipage}{2.5cm}
${\textnormal{error}}(s)$


$s \leftarrow o(x^2) $

\end{minipage} &
\begin{minipage}{7.5cm}

\begin{eqnarray}\nonumber
  &&|s-x^2| \\\nonumber
  &       \leq&  \ulp(s)\;\;(\star)\\\nonumber
\end{eqnarray}


\end{minipage} &
\begin{minipage}{6cm}

\end{minipage}\\%\hline
\begin{minipage}{2.5cm}
${\textnormal{error}}(t)$


$t \leftarrow \minf(s+1) $
$(\bullet)$
\end{minipage} &
\begin{minipage}{7.5cm}



\begin{eqnarray}\nonumber
  &&|t-(x^2+1)| \\\nonumber
  &       \leq&  2 \ulp(t) \;\;(\star)\\\nonumber
\end{eqnarray}


\end{minipage} &
\begin{minipage}{6cm}

($\star$)

see subsection \ref{generic:sous}


\end{minipage}\\%\hline
\begin{minipage}{2.5cm}
${\textnormal{error}}(u)$


$u \leftarrow o(\sqrt{t}) $


\end{minipage} &
\begin{minipage}{7.5cm}

\begin{eqnarray}\nonumber
  &&|u-\sqrt{x^2+1}| \\\nonumber
  &       \leq& 3 \ulp(u) \;(\star)
\end{eqnarray}


\end{minipage} &
\begin{minipage}{6cm}

($\star$)

see subsection \ref{generic:sqrt}

with ($\bullet$)

\end{minipage}\\%\hline
\begin{minipage}{2.5cm}
${\textnormal{error}}(v)$


$v \leftarrow o(u+x) $


\end{minipage} &
\begin{minipage}{7.5cm}

\begin{eqnarray}\nonumber
  &&|v-(\sqrt{x^2+1}+x)| \\\nonumber
  &       \leq& 5 \ulp(v) \;(\star)
\end{eqnarray}


\end{minipage} &
\begin{minipage}{6cm}

($\star$)

see subsection \ref{generic:sous}

\end{minipage}\\%\hline
\begin{minipage}{2.5cm}
${\textnormal{error}}(w)$


$w \leftarrow o(\log v) $
\end{minipage} &
\begin{minipage}{7.5cm}

\begin{eqnarray}\nonumber
  &&|w-\log(\sqrt{x^2+1}+x)| \\\nonumber
  &       \leq& (1+5.2^{1-\Exp(w)}) \ulp(w) \;\star
\end{eqnarray}


\end{minipage} &
\begin{minipage}{6cm}

($\star$)

see subsection \ref{generic:log}

\end{minipage}
\end{tabular}
\end{center}

That shows the rounding error on the calculation of $\n{asinh} x$ can
be bound by $ (1+5.2^{1-\Exp(w)})\;\; \ulp$ on the result. So, to
calculate the size of intermediary variables, we have to add, at
least, $\lceil \log_2 (1+5.2^{1-\Exp(w)}) \rceil$ bits the wanted
precision.
 
\subsection{The hyperbolic tangent function}

The {\tt mpfr\_tanh} ($\tanh{x}$) function implements the hyperbolic
tangent as :

$$
\tanh x = \frac{ e^{2 \cdot x} -1 }{ e^{2 \cdot x} +1} 
$$

The algorithm used for the calculation of the hyperbolic tangent is as follows\footnote{$o()$ represent the arrondi error and $error(u)$ the
  error associate with the calcualtion of $u$}:

\begin{eqnarray}\nonumber
u&\leftarrow&o(2 \cdot x)\\\nonumber
v&\leftarrow&o(e^u)\\\nonumber
w&\leftarrow&o(v+1)\\\nonumber
r&\leftarrow&o(v-1)\\\nonumber
s&\leftarrow&o(\frac{r}{w})
\end{eqnarray}

Now, we have to bound the rounding error for each step of this
algorithm.  First, let consider the parity of hyperbolic tangent
($\tanh(-x)=-\tanh(x)$) : the problem is reduced to calculate $\tanh x$ with $x \geq 0$. We can deduce $e^x \geq 1$.


\begin{center}
\begin{tabular}{l l l}

\begin{minipage}{2.5cm}


${\textnormal{error}}(u)$


$u \leftarrow o(2 \cdot x)$


\end{minipage} &
\begin{minipage}{7.5cm}


exact

\end{minipage} &
\begin{minipage}{6cm}
{\hspace{7cm}}
\end{minipage}\\%\hline
\begin{minipage}{2.5cm}
${\textnormal{error}}(v)$


$v \leftarrow o(e^{u}) $

\end{minipage} &
\begin{minipage}{7.5cm}



\begin{eqnarray}\nonumber
  &&|v-e^{2 \cdot x}| \\\nonumber
  &       \leq&  |v - e^{u}| +  |e^{u}  - e^{2 \cdot x}|\\\nonumber
  &       \leq& \ulp(v)
\end{eqnarray}


\end{minipage} &
\begin{minipage}{6cm}


\end{minipage}\\%\hline
\begin{minipage}{2.5cm}
${\textnormal{error}}(w)$


$w \leftarrow \minf(v+1) $

$(\bullet)$
\end{minipage} &
\begin{minipage}{7.5cm}



\begin{eqnarray}\nonumber
  &&|w-(e^{2 \cdot x}+1)| \\\nonumber
  &\leq&  |w - (v+1)|+|(v+1) - (e^{2 \cdot x}+1)|\\\nonumber
  &\leq&  \ulp(w) + \ulp(v)\\\nonumber
  &\leq&  2 \cdot \ulp(w)\;(\star)\\\nonumber
\end{eqnarray}

\end{minipage} &
\begin{minipage}{6cm}
$(\star)$

With $v \leq w$ then $\ulp(v) \leq \ulp(w)$ 

\end{minipage}\\%\hline
\begin{minipage}{2.5cm}
${\textnormal{error}}(r)$


$r \leftarrow \pinf(v-1) $

$(\bullet\bullet)$
\end{minipage} &
\begin{minipage}{7.5cm}



\begin{eqnarray}\nonumber
  &&|r-(e^{2 \cdot x}-1)| \\\nonumber
  &\leq&  |r - (v-1)|+|(v-1) - (e^{2 \cdot x}-1)|\\\nonumber
  &\leq&  \ulp(r) + \ulp(v)\\\nonumber
  &\leq&  (1+2^{\Exp(v)-\Exp(r)}) \ulp(r)\;(\star)\\\nonumber
\end{eqnarray}

\end{minipage} &
\begin{minipage}{6cm}
$(\star)$


see subsection \ref{generic:sous}

\end{minipage}\\%\hline
\begin{minipage}{2.5cm}
${\textnormal{error}}(s)$


$s \leftarrow o(\frac{r}{w}) $
\end{minipage} &
\begin{minipage}{7.5cm}



\begin{eqnarray}\nonumber
  &&|s-\frac{e^{2x}-1}{e^{2x}+1}| \\\nonumber
  &\leq& (1+2 \times 2+ \hdots\\\nonumber
  &\hdots& 2(1+2^{\Exp(v)-\Exp(r)})) \ulp(s) \;\;(\star)\\\nonumber
\end{eqnarray}

\end{minipage} &
\begin{minipage}{6cm}
$(\star)$


see subsection \ref{generic:div}


with $(\bullet)$ and $(\bullet\bullet)$

\end{minipage}
\end{tabular}
\end{center}


That show the rounding error on the calculation of $\tanh x$ can be
bound by $(1+2 \times 2+2(1+2^{\Exp(v)-\Exp(r)})) \ulp(s)$, then the
number of bits need to add to the want accuracy to define intermediary
variable is :

\[
N_t=\lceil \log_2(7+2^{\Exp(v)-\Exp(r)+1}) \rceil
\]





\subsection{The inverse hyperbolic tangent function}

The {\tt mpfr\_atanh} ($\n{acosh}{x}$) function implements the inverse hyperbolic tangent as :

$$
\n{atanh} = \frac{1}{2} \log \frac{1+x}{1-x} 
$$

The algorithm used for the calculation of the inverse hyperbolic tangent is as follows

\begin{eqnarray}\nonumber
s&\leftarrow&o(1+x)\\\nonumber
t&\leftarrow&o(1-x)\\\nonumber
u&\leftarrow&o(\frac{s}{t})\\\nonumber
v&\leftarrow&o(\log u)\\\nonumber
w&\leftarrow&o(\frac{1}{2} v)
\end{eqnarray}


Now, we have to bound the rounding error for each step of this
algorithm. First, let consider the parity of hyperbolic arctan
($\n{atanh}(-x)=-\n{atanh}(x)$) : the problem is reduced to calculate
$\n{atanh} x$ with $x \geq 0$.

\begin{center}
\begin{tabular}{l l l}

\begin{minipage}{2.5cm}
${\textnormal{error}}(s)$


$s \leftarrow \pinf(1+x) $
$(\bullet)$
\end{minipage} &
\begin{minipage}{7.5cm}

\begin{eqnarray}\nonumber
  &&|s-(1+x)| \\\nonumber
  &       \leq&  2 \ulp(s)\;\;(\star)\\\nonumber
\end{eqnarray}


\end{minipage} &
\begin{minipage}{6cm}

see subsection \ref{generic:sous}


\end{minipage}\\%\hline
\begin{minipage}{2.5cm}
${\textnormal{error}}(t)$

$t \leftarrow \minf(1-x) $
$(\bullet\bullet)$
\end{minipage} &
\begin{minipage}{7.5cm}



\begin{eqnarray}\nonumber
  &&|t-(1-x)| \\\nonumber
  &       \leq&  (1+2^{\Exp(x)-\Exp(t)}) \ulp(t) \;\;(\star)\\\nonumber
\end{eqnarray}


\end{minipage} &
\begin{minipage}{6cm}

($\star$)

see subsection \ref{generic:sous}


\end{minipage}\\%\hline
\begin{minipage}{2.5cm}
${\textnormal{error}}(u)$


$u \leftarrow o(\frac{s}{t}) $


\end{minipage} &
\begin{minipage}{7.5cm}

\begin{eqnarray}\nonumber
  &&|u-\frac{1+x}{1-x}| \\\nonumber
  &       \leq& (1+ 2 \times 2 + \\\nonumber
  &       \cdots& 2 \times (1+2^{\Exp(x)-\Exp(t)}))\ulp{u} \;(\star)\\\nonumber
  &       \leq& (7+2^{\Exp(x)-\Exp(t)+1})\ulp(u)
\end{eqnarray}


\end{minipage} &
\begin{minipage}{6cm}

($\star$)

see subsection \ref{generic:inv}

with ($\bullet$) and ($\bullet\bullet$)

\end{minipage}\\%\hline
\begin{minipage}{2.5cm}
${\textnormal{error}}(v)$


$v \leftarrow o(\log(u)) $


\end{minipage} &
\begin{minipage}{7.5cm}

\begin{eqnarray}\nonumber
  &&|v-(\log{\frac{1+x}{1-x}})| \\\nonumber
  &       \leq& (1+(7+2^{\Exp(x)-\Exp(t)+1}) \\\nonumber
  &       \cdots&  \times 2^{1-\Exp(v)}) \ulp(v)\;(\star)\\\nonumber
  &       \leq& (1+7 \times 2^{1-\Exp(v)} +\\\nonumber
  &       \cdots&  2^{\Exp(x)-\Exp(t)-\Exp(v)+2}) \ulp(v) 
\end{eqnarray}


\end{minipage} &
\begin{minipage}{6cm}

($\star$)

see subsection \ref{generic:log}

\end{minipage}\\%\hline
\begin{minipage}{2.5cm}
${\textnormal{error}}(w)$


$w \leftarrow o(\frac{1}{2} v) $
\end{minipage} &
\begin{minipage}{7.5cm}

\begin{eqnarray}\nonumber
  &&|w-\frac{1}{2}\log{\frac{1+x}{1-x}}| \\\nonumber
  &       \leq& (1+7 \times 2^{1-\Exp(v)} + \\\nonumber
  &       \cdots&  2^{\Exp(x)-\Exp(t)-\Exp(v)+2}) \ulp(w) \;\star
\end{eqnarray}


\end{minipage} &
\begin{minipage}{6cm}

($\star$) exact


\end{minipage}
\end{tabular}
\end{center}

That shows the rounding error on the calculation of $\n{atanh} x$ can
be bound by $ (1+7 \times 2^{1-\Exp(v)} +
2^{\Exp(x)-\Exp(t)-\Exp(v)+2})\;\; \ulp$ on the result. So, to
calculate the size of intermediary variables, we have to add, at
least, $\lceil \log_2 (1+7 \times 2^{1-\Exp(v)} +
2^{\Exp(x)-\Exp(t)-\Exp(v)+2}) \rceil$ bits the wanted precision.
 
\subsection{The arc-sine function}

\begin{enumerate}
\item We use the formula $arcsin\,x=\arctan\,\frac{x}{\sqrt{1-x^2}}$
\item We will have the when $x$ is near $1$ we will experience uncertainty problems:
\item If $x=a(1+\delta)$ with $\delta$ being the relative error then we will have
\begin{equation*}
1-x=1-a-a\delta=(1-a)[1-\frac{a}{1-a}\delta]
\end{equation*}
Ans so when using the arctangent programs we need to take into account that decrease in precision.
\item We will have 
\end{enumerate}

\subsection{The arc-cosine function} % from Mathieu Dutour

\begin{enumerate}
\item Obviously, we used the formula 
\begin{equation*}
\arccos\,x=\frac{\pi}{2}-\arcsin\,x
\end{equation*}
\item The problem of $\arccos$ is that it is $0$ at $1$, so, we have a cancellation problem to treat at $1$.
\item (Suppose $x\geq 0$, this is where the problem happens) The derivative of $\arccos$ is $\frac{-1}{\sqrt{1-x^2}}$ and we will have
\begin{equation*}
\frac{1}{2\sqrt{1-x}}  \leq   |\frac{-1}{\sqrt{1-x^2}}|=\frac{1}{\sqrt{(1-x)(1+x)}}  \leq  \frac{1}{\sqrt{1-x}}
\end{equation*}
So, integrating the above inequality on $[x,1]$ we get
\begin{equation*}
\sqrt{1-x}\leq \arccos\,x\leq 2\sqrt{1-x}
\end{equation*}
\item The important part is the lower bound that we get which tell us a upper bound on the cancellation that will occur:\\
The terms that are cancelled are $\pi/2$ and $\arcsin\,x$, their order is $2$. The number of canceled terms is so
\begin{verbatim}
1-1/2*MPFR_EXP(1-x)
\end{verbatim}
\end{enumerate}

\subsection{The euclidean distance function}

The {\tt mpfr\_hypot} ($\textnormal{hypot}(x,y)$) function implements the euclidean distance function  as :

\[
\textnormal{hypot} (x,y) = \sqrt{x^2+y^2}
\]

The algorithm used for the calculation of the euclidean distance is as follows:

\begin{eqnarray}\nonumber
u&\leftarrow&o(x^2)\\\nonumber
v&\leftarrow&o({y}^{2})\\\nonumber
w&\leftarrow&o(u+v)\\\nonumber
s&\leftarrow&o(\sqrt{w})
\end{eqnarray}

Now, we have to bound the rounding error for each step of this
algorithm.  



\begin{center}
\begin{tabular}{l l l}

\begin{minipage}{2.5cm}


${\textnormal{error}}(u)$


$u \leftarrow o(x^2)$

\end{minipage} &
\begin{minipage}{7.5cm}

\begin{eqnarray}\nonumber
  |u-x^2| &\leq& ulp(u)\\\nonumber
\end{eqnarray}

\end{minipage} &
\begin{minipage}{6cm}
{\hspace{7cm}}
\end{minipage}\\%\hline
\begin{minipage}{2.5cm}
${\textnormal{error}}(v)$


$v \leftarrow o({y}^{2}) $

\end{minipage} &
\begin{minipage}{7.5cm}

\begin{eqnarray}\nonumber
  |v-y^2| &\leq& ulp(v)\\\nonumber
\end{eqnarray}


\end{minipage} &
\begin{minipage}{6cm}


\end{minipage}\\%\hline
\begin{minipage}{2.5cm}
${\textnormal{error}}(w)$


$w \leftarrow \minf(u+v) $\\
$(\bullet)$
\end{minipage} &
\begin{minipage}{7.8cm}



\begin{eqnarray}\nonumber
  &&|w-(x^2+y^2)| \\\nonumber
  &       \leq&  |w - (u+v)|+|u - x^2|+|v - y^2|\\\nonumber
  &       \leq& \ulp(w) + \ulp(u) + \ulp(v)\\\nonumber
  &       \leq& 3\ulp(w)\;\;(\star)\\\nonumber
\end{eqnarray}


\end{minipage} &
\begin{minipage}{6cm}

$(\star)$

With $v \leq w$ and $u \leq w$

\end{minipage}\\%\hline
\begin{minipage}{2.5cm}
${\textnormal{error}}(s)$

$s \leftarrow o(\sqrt{w}) $

\end{minipage} &
\begin{minipage}{7.5cm}

\begin{center}



\begin{eqnarray}\nonumber
  &&|s-(\sqrt{x^2+y^2})| \\\nonumber
  &\leq& (1+\frac{2 \times  3}{1+\sqrt{1}}) \ulp(s) \;\; (\star)\\\nonumber
   &\leq& 4 \ulp(s) 
\end{eqnarray}



\end{center}

\end{minipage} &
\begin{minipage}{6cm}

$(\star)$
see subsection \ref{generic:sqrt}

with $c_w^- = 1$ 

for $w \leftarrow \minf(u+v)$ ($\bullet$)

\end{minipage}


\end{tabular}
\end{center}



That shows the rounding error on the calculation of $\sqrt{x^2+y^2}$ can be
bound by $4 \, \ulp$ on the result. So, to calculate the size of
intermediary variables, we have to add, at least, $\lceil \log_2 4 \rceil=2$ bits the wanted precision.


\subsection{The floating multiply-add function}

The {\tt mpfr\_fma} ($\n{fma}(x,y,z)$) function implements the floating multiply-add function  as :

\[
\textnormal{fma} (x,y,z) = z + x \times y
\]

The algorithm used for this calculation is as follows:

\begin{eqnarray}\nonumber
u&\leftarrow&o(x \times y)\\\nonumber
v&\leftarrow&o(z + u)\\\nonumber
\end{eqnarray}

Now, we have to bound the rounding error for each step of this
algorithm.  



\begin{center}
\begin{tabular}{l l l}

\begin{minipage}{2.5cm}


${\textnormal{error}}(u)$


$u \leftarrow o(x \times y)$

\end{minipage} &
\begin{minipage}{7.5cm}

\begin{eqnarray}\nonumber
  |u-(xy)| &\leq& ulp(u)\\\nonumber
\end{eqnarray}

\end{minipage} &
\begin{minipage}{6cm}
{\hspace{7cm}}
\end{minipage}\\%\hline
\begin{minipage}{2.5cm}
${\textnormal{error}}(v)$


$v \leftarrow o(z+u) $

\end{minipage} &
\begin{minipage}{7.5cm}

\begin{eqnarray}\nonumber
  |v-(z+xy)| &\leq& \ulp(v) + |(z+u) - (z+xy)|\\\nonumber
&\leq& (1+2^{e_u-e_v})\ulp(v)\;\;(\star)\\\nonumber
\end{eqnarray}


\end{minipage} &
\begin{minipage}{6cm}
($\star$)

see subsection \ref{generic:sous}


\end{minipage}
\end{tabular}
\end{center}



That shows the rounding error on the calculation of $\n{fma}(x,y,z)$ can be
bound by $(1+2^{e_u-e_v}) \ulp$ on the result. So, to calculate the size of
intermediary variables, we have to add, at least, $\lceil \log_2 (1+2^{e_u-e_v})\rceil$ bits the wanted precision.

\subsection{The expm1 function}

The {\tt mpfr\_expm1} ($\n{expm1}(x)$) function implements the expm1 function  as :

\[
\textnormal{expm1} (x) = e^x -1
\]

The algorithm used for this calculation is as follows:

\begin{eqnarray}\nonumber
u&\leftarrow&o(e^x)\\\nonumber
v&\leftarrow&o(u-1)\\\nonumber
\end{eqnarray}

Now, we have to bound the rounding error for each step of this
algorithm.  


\begin{center}
\begin{tabular}{l l l}

\begin{minipage}{2.5cm}


${\textnormal{error}}(u)$


$u \leftarrow o(e^x)$

\end{minipage} &
\begin{minipage}{7.5cm}

\begin{eqnarray}\nonumber
  |u-e^x| &\leq& ulp(u)\\\nonumber
\end{eqnarray}

\end{minipage} &
\begin{minipage}{6cm}
{\hspace{7cm}}
\end{minipage}\\%\hline
\begin{minipage}{2.5cm}
${\textnormal{error}}(v)$


$v \leftarrow o(u-1) $

\end{minipage} &
\begin{minipage}{7.5cm}

\begin{eqnarray}\nonumber
  |v-(e^x-1)| &\leq& (1+2^{e_u-e_v})\ulp(v)\;\;(\star)
\end{eqnarray}


\end{minipage} &
\begin{minipage}{6cm}
($\star$)

see subsection \ref{generic:sous}


\end{minipage}
\end{tabular}
\end{center}



That shows the rounding error on the calculation of $\n{expm1}(x)$ can be
bound by $(1+2^{e_u-e_v}) \ulp$ on the result. So, to calculate the size of
intermediary variables, we have to add, at least, $\lceil \log_2 (1+2^{e_u-e_v})\rceil$ bits the wanted precision.

\subsection{The log1p function}

The {\tt mpfr\_log1p} ($\n{log1p}(x)$) function implements the log1p function  as :

\[
\textnormal{log1p} (x) = \log(1+x)
\]

The algorithm used for this calculation is as follows:

\begin{eqnarray}\nonumber
u&\leftarrow&o(x)\\\nonumber
v&\leftarrow&o(1+u)\\\nonumber
w&\leftarrow&o(\log(v))\\\nonumber
\end{eqnarray}

Now, we have to bound the rounding error for each step of this
algorithm.  

\begin{center}
\begin{tabular}{l l l}

\begin{minipage}{2.5cm}


${\textnormal{error}}(u)$


$u \leftarrow o(x)$

\end{minipage} &
\begin{minipage}{7.5cm}

\begin{eqnarray}\nonumber
  |u-x| &\leq& ulp(u)\\\nonumber
\end{eqnarray}

\end{minipage} &
\begin{minipage}{6cm}
{\hspace{7cm}}
\end{minipage}\\%\hline
\begin{minipage}{2.5cm}
${\textnormal{error}}(v)$


$v \leftarrow o(1+u) $

\end{minipage} &
\begin{minipage}{7.5cm}

\begin{eqnarray}\nonumber
  |v-(1+x)| &\leq& (1+2^{e_u-e_v})\ulp(v)
\end{eqnarray}


\end{minipage} &
\begin{minipage}{6cm}
($\star$)
see subsection \ref{generic:sous}
\end{minipage}\\%\hline
\begin{minipage}{2.5cm}
${\textnormal{error}}(w)$


$w \leftarrow o(\log(v)) $

\end{minipage} &
\begin{minipage}{7.5cm}
\begin{eqnarray}\nonumber
  |v-\log{v}| &\leq& (1+(1+2^{e_u-e_v})2^{1-e_w})\\\nonumber
 &\hdots& \ulp(w)
\end{eqnarray}
\end{minipage} &
\begin{minipage}{6cm}
($\star$)
see subsection \ref{generic:log}
\end{minipage}
\end{tabular}
\end{center}


That shows the rounding error on the calculation of $\log\n{1p}(x)$
can be bound by $(1+(1+2^{e_u-e_v})2^{1-e_w}) \ulp$ on the result. So,
to calculate the size of intermediary variables, we have to add, at
least, $\lceil \log_2 (1+(1+2^{e_u-e_v})2^{1-e_w}) \rceil$ bits the wanted
precision.


\subsection{The log2 or log10 function}

The {\tt mpfr\_log2} or {\tt mpfr\_log10} function implements the log in base 2 or 10 function  as :

\[
\textnormal{log2} (x) = \frac{log{x}}{log{2}}
\]

or

\[
\textnormal{log10} (x) = \frac{log{x}}{log{10}}
\]


The algorithm used for this calculation is the same for $\n{log2}$ or
$\n{log10}$ and is descibed as follows for $t=2 \n{ or } 10$:

\begin{eqnarray}\nonumber
u&\leftarrow&o(\log(x))\\\nonumber
v&\leftarrow&o(\log(t))\\\nonumber
w&\leftarrow&o(\frac{u}{v})\\\nonumber
\end{eqnarray}

Now, we have to bound the rounding error for each step of this
algorithm with $x \geq 0$ and $y$ is a floating number.

\begin{center}
\begin{tabular}{l l l}

\begin{minipage}{2.5cm}


${\textnormal{error}}(u)$


$u \leftarrow \pinf(\log(x))$
$(\bullet)$
\end{minipage} &
\begin{minipage}{7.5cm}

\begin{eqnarray}\nonumber
  |u-\log(x)| &\leq& \ulp(u)
\end{eqnarray}

\end{minipage} &
\begin{minipage}{6cm}

\end{minipage}\\%\hline
\begin{minipage}{2.5cm}
${\textnormal{error}}(v)$


$v \leftarrow \minf(\log{t}) $
$(\bullet\bullet)$
\end{minipage} &
\begin{minipage}{7.5cm}

\begin{eqnarray}\nonumber
  |v-{\log t}| &\leq& \ulp(v)
\end{eqnarray}

\end{minipage} &
\begin{minipage}{6cm}
\end{minipage}\\%\hline
\begin{minipage}{2.5cm}
${\textnormal{error}}(w)$


$w \leftarrow \o(\frac{u}{v}) $

\end{minipage} &
\begin{minipage}{7.5cm}

\begin{eqnarray}\nonumber
  |v-(\frac{\log x}{\log t})| &\leq& 5 \ulp(w) \;\;(\star)
\end{eqnarray}


\end{minipage} &
\begin{minipage}{6cm}
($\star$)

see subsection \ref{generic:div}

\end{minipage}

\end{tabular}
\end{center}


That shows the rounding error on the calculation of $log2 \n{ or }
log10$ can be bound by $ 5 \ulp$ on the result. So, to calculate the
size of intermediary variables, we have to add, at least, 3 bits the
wanted precision.


\subsection{The power function}

The {\tt mpfr\_pow} ($\n{pow}(x,y)$) function implements the power function  as :

\[
\textnormal{pow} (x,y) = e^{y \log(x)}
\]

The algorithm used for this calculation is as follows:

\begin{eqnarray}\nonumber
u&\leftarrow&o(\log(x))\\\nonumber
v&\leftarrow&o(y \times u)\\\nonumber
w&\leftarrow&o(e^v)\\\nonumber
\end{eqnarray}

Now, we have to bound the rounding error for each step of this
algorithm with $x \geq 0$ and $y$ is a floating number.

\begin{center}
\begin{tabular}{l l l}

\begin{minipage}{2.5cm}


${\textnormal{error}}(u)$


$u \leftarrow o(\log(x))$

\end{minipage} &
\begin{minipage}{7.5cm}

\begin{eqnarray}\nonumber
  |u-\log(x)| &\leq& \ulp(u)\;\;\star
\end{eqnarray}

\end{minipage} &
\begin{minipage}{6cm}

\end{minipage}\\%\hline
\begin{minipage}{2.5cm}
${\textnormal{error}}(v)$


$v \leftarrow \pinf(y \times u) $
($\bullet$)
\end{minipage} &
\begin{minipage}{7.5cm}

\begin{eqnarray}\nonumber
  |v-(y\log(x))| &\leq& \ulp(v) + |y.u-y.\log(x)|\\\nonumber
   &\leq& \ulp(v) + y|u-\log(x)|\\\nonumber
   &\leq& \ulp(v) + y \ulp(u)\\\nonumber
   &\leq& \ulp(v) + 2 \ulp(yu) \;(\star)\\\nonumber
   &\leq& 3 \ulp(v)  \;(\star\star)
\end{eqnarray}


\end{minipage} &
\begin{minipage}{6cm}
($\star$)

with \U{R4}

($\star$)

with \U{R8}

\end{minipage}\\%\hline
\begin{minipage}{2.5cm}
${\textnormal{error}}(w)$


$w \leftarrow o(e^v) $

\end{minipage} &
\begin{minipage}{7.5cm}
\begin{eqnarray}\nonumber
  |w-e^v| &\leq& (1+3.2^{\Exp(v)+1}) \ulp(w)
\end{eqnarray}
\end{minipage} &
\begin{minipage}{6cm}
($\star$)
see subsection \ref{generic:exp}

with $c_u^* = 1$ for $(\bullet)$
\end{minipage}
\end{tabular}
\end{center}


That shows the rounding error on the calculation of $x^y$
can be bound by $ (1+3.2^{\Exp(v)+1})\;\; \ulp$ on the result. So,
to calculate the size of intermediary variables, we have to add, at
least, $\lceil \log_2 (1+3.2^{\Exp(v)+1}) \rceil$ bits the wanted
precision.



\subsection{The real cube root}

The {\tt mpf\_cbrt} ($\sqrt[3]{x}$) function implements a Newton
algorithm to calculate the real cube root of $x$. {\bf Note:} The case
$x<0$ will not be considered, the function {\tt mpf\_cbrt} returns $-\sqrt[3](-x)$ if $x<0$.

\subsubsection{Newton algorithm applied on cubic}

In general case, to converge iterativelly to a
root of the equation $f(x)=0$, we can use a Newton algoritm which is describe by the formulation :
$x_{k+1}=x_k-\frac{f(x_k)}{f^{'}(x_k)}$.

To applied this algorithm to calculate the real cubic root of
$N,\;\;: \sqrt[3]{N}$, we want to solve the equation $f(x)=x^3-N=0$. The
Newton algorithm become :

\begin{equation}\label{NewtonCbrt}
x_{k+1}=\frac{1}{3}[2x_k+\frac{N}{x^2_k}]
\end{equation}

{\bf Note :} The value of $x_0$ will be discuss after.

\subsubsection{Number representation}

The MPFR natural representatio of a number is: $N=m \times 2^e$, with
$m$ the mantissa ($\frac{1}{2} \leq m < 1$) and $e$ the exponent.

To simplify the implantation of this real cube root algorithm, we
choose to represent the exponent of $N$ by $e=3 \times e^{'}-r$ where $r \in {0,1,2}$.
In this condition the representation of $N$ become:
$\frac{m}{2^r} \times 2^{3 e^{'}}$. Now, we put $m^{'}=\frac{m}{2^r}$
with $\frac{1}{8} \leq m^{'} < 1$.


Calculate $\sqrt[3]{m \times 2^e} = \sqrt[3]{\frac{m}{2^r} \times 2^{3 e^{'}}}$, that is egal to
compute $\sqrt[3]{\frac{m}{2^r}} \times 2^{e^{'}}$.


So, the calculation of $\sqrt[3]{N}$ is reduce to the calculation of
$\sqrt[3]{m^{'}}$.

\subsubsection{Newton iteration error}

By considering the simplifications introduce on the section before, we
want to produice the error $\epsilon_{k+1}$ between the calculation of the real cube
root of $N, \;\; \frac{1}{8}\leq N < 1$ at the iteration $k+1$ of the
nwton algorithm, noted $x_{k+1}$, and the real value of
$\sqrt[3]{N}$. First, we want to express this error as a function of
the error $\epsilon_{k}$ at tke iteration $k$.

\begin{equation}\label{ErrorCbrt}
\epsilon_{k+1}= |x_{k+1}-N^{\frac{1}{3}}|
\end{equation}

By using, equation \ref{NewtonCbrt} and \ref{ErrorCbrt} we obtain :

\begin{eqnarray}\nonumber
\epsilon_{k+1}&=&   \arrowvert\frac{1}{3}[2x_k+\frac{N}{x^2_k}]-N^{\frac{1}{3}} \arrowvert\\\nonumber
  &=& \arrowvert (x_k-N^{\frac{1}{3}})-\frac{1}{3x^2_k}[x^3_k-N] \arrowvert \\\nonumber
  &=& \arrowvert
  \frac{x_k-N^{\frac{1}{3}}}{3x^2_k}[(x_k-N^{\frac{1}{3}})(x_k+N^{\frac{1}{3}})+x_k(x_k-N^{\frac{1}{3}})] \arrowvert\\\nonumber
  &=&\arrowvert \frac{(x_k-N^{\frac{1}{3}})^2}{3x^2_k}
  [2x_k+N^{\frac{1}{3}}]\arrowvert\\\nonumber
  &=&  \epsilon^2_{k}\arrowvert \frac{2x_k+N^{\frac{1}{3}}}{3x^2_k}\arrowvert\\\nonumber
\end{eqnarray}

To bound this expression, we put $x_0=\frac{3}{4}$. If $x_k >0 $, with
$\frac{1}{8}\leq N <1$, $\frac{2x_k+N^{\frac{1}{3}}}{3x^2_k} >0 $ and
then $\epsilon_{k+1} >0$ in othe hand $ N^{\frac{1}{3}}\leq x_{k+1}$,
we can obtain this inegality for $1 \leq k$:

\begin{equation}
N^{\frac{1}{3}}\leq x_{k}
\end{equation}


We can now bound $\epsilon_{k+1}$ :

\begin{eqnarray}\nonumber
\epsilon_{k+1} =  \epsilon^2_{k}
\frac{2x_k+N^{\frac{1}{3}}}{3x^2_k} \leq
\epsilon^2_{k}\frac{2x_k+x_k}{3x^2_k}\leq 2 \epsilon^2_{k}\\\nonumber
\end{eqnarray}

Because $\frac{1}{2}\leq<N^{\frac{1}{3}}$, $2\geq\frac{1}{x_k}$



\subsubsection{Size of tempory variables}


First, $z_k$ is the MPFR calculation approximation of $x_k$. The error
produce by the calculation of $z_k$ at the iteration $k$ is bounded by
:
\[
|z_k-x_k| \leq n_k \times ulp(z_k)
\]

with $n_k$ the number of ulp()...a completer

The algorithm used to calculate one Newton iteration is:

\begin{eqnarray}\nonumber
t&\leftarrow&2z_k\\\nonumber
u&\leftarrow&z_k^2\\\nonumber
v&\leftarrow&\frac{N}{u}\\\nonumber
w&\leftarrow&t+v\\\nonumber
z_{k+1}&\leftarrow&\frac{w}{3}\\\nonumber
\end{eqnarray}

Let describe the additionnal error produce at each step of the
algorithm :

\[
t \leftarrow 2 z_k
\]

The error associate with the calculation of $t$ can be bound as
following:

\begin{eqnarray}\nonumber
  error(t) &\leq& 2 n_k ulp(z_k)\\\nonumber
         &\leq& n_k ulp(t)\\\nonumber
\end{eqnarray}


\[
u\leftarrow z_k^2
\]



\begin{eqnarray}\nonumber
  error(u) = |o(z_k^2)-x_k^2| &\leq& |o(z_k^2)-z_k^2| +
  |z_k^2-x_k^2|\\\nonumber
         &\leq& 1 ulp(z_k^2) + n_k ulp(1) (z_k+x_k)\\\nonumber
         &\leq& (2 n_k + 1) ulp(1)\\\nonumber
\end{eqnarray}

\[
v \leftarrow \frac{N}{u}
\]

\begin{eqnarray}\nonumber
  error(v) = |o(\frac{N}{u})-\frac{N}{x_k^2}| &\leq& ulp(v) +
  |\frac{N}{u}-\frac{N}{x_k^2}|\\\nonumber
         &\leq& 4 ulp(1) + \frac{N}{u
           x_k^2}|x_k^2-u|\;\;\;{\textnormal{with}} \frac{1}{8}\leq v <4\\\nonumber
         &\leq& 4 ulp(1) + 4 |x_k^2-u|\;\;\;{\textnormal{with}} \;\;N^{\frac{1}{3}} \leq x_k\;\;\frac{N}{u
           x_k^2} \leq \frac{x_k^3}{u x_k^2}\leq 4
         \\\nonumber
         &\leq& 4 ulp(1) + 4(2 n_k + 1) ulp(1)\\\nonumber
         &\leq& (8 n_k + 8) ulp(1)\\\nonumber
\end{eqnarray}


\[
w \leftarrow t+v
\]

\begin{eqnarray}\nonumber
  error(w) = |o(t+v)-(2x_k + \frac{N}{u})| &\leq& ulp(w) +
  |v-\frac{N}{u}|+|t-2x_k|\\\nonumber
  &\leq& 2ulp(1) +(2 n_k + 1) ulp(1)  + (8 n_k + 8)
  ulp(1)\;\;\;{\textnormal{with ... demander Paul}}\\\nonumber
  &\leq& (10 n_k + 11) ulp(1)\\\nonumber  
\end{eqnarray}

\[
z_{k+1} \leftarrow \frac{w}{3}
\]

\begin{eqnarray}\label{errsize}
  error(z_{k+1}) = |o(\frac{w}{3})-\frac{1}{3}[2x_k + \frac{N}{u}]| &\leq& ulp(z_{k+1}) +
  \frac{1}{3}[(10 n_k + 11) ulp(1)]\\\nonumber  
&\leq& ulp(2) + \frac{1}{3}[(10 n_k + 11) ulp(1)]\\\nonumber  
  &\leq& \frac{10 n_k + 17}{3} ulp(1)\\\nonumber  
\end{eqnarray}



\subsubsection{Implentation}

First, we have to calculate the number of iteration of Newton's
algorithme need to garanty the wanted precision of the result with $n$ exact bits.

The section "Newton iteration error" show how we can calculate the
error between two Newton iterations : we obtain $\epsilon_{k+1} \leq 2
\epsilon_k^2$.  If $x_0=\frac{3}{4}$ we get :

\[
\epsilon_{0}=|x_0-N^{\frac{1}{3}}|\leq\frac{1}{4}\;\;{\textnormal{With  }}N^{\frac{1}{3}} \geq\frac{1}{2}
\]

then, $\epsilon_1 \leq \frac{1}{8}$, $\epsilon_2 \leq \frac{1}{32}$, ....


To certifiate result with $n$ bits of precision, we have to get $\epsilon_k\leq\frac{1}{2^{n}}$, so $n\geq\tau_k$, with $\tau_k=-\log_2 \epsilon_k$.

To calculate $k$ the number of iteration need to certifiate result, we have the compute $\tau_0=2,\, \tau_1=3,\, \tau_{k}=2\tau_{k-1} -1$ such as $\tau_k \leq n$.

For each Newton iteration the intermediate calculation give an error bound by inequality \ref{errsize}. With the knowledge of $k$, for an intitial error null $n_0=0$, we can compute the error for all Newton iterations : $n_{k} = \frac{10 n_{k-1}+14}{3}$. Then, the size in bits $m$ of the precision of the intermediary variables is give by :

\[
m > n + \log_2 n_k + \log_2 n
\]

\begin{table}
\begin{center}
\begin{tabular}{|c|c|c|} \hline
   $w$        & $\err(w)/\ulp(w) \le c_w + \ldots$ &special case\\ \hline\hline
$o(u+v)$ & $k_u 2^{e_u-e_w} + k_v 2^{e_v-e_w}$ & $k_u + k_v$ if $u v \ge 0$\\
$o(u \cdot v)$ & $(1+c^{+}_u)k_u + (1+c^{+}_v)k_v$ & $2k_u + 2k_v$ if $u \ge x$, $v \ge y$\\
$o(1/v)$ & $4 k_v$ & $2 k_v$ if $v \le y$ \\
$o(u/v)$ & $4 k_u + 4 k_v$ & $2 k_u + 2 k_v$ if $v \le y$ \\
$o(\sqrt{u})$ & $2 k_u/(1+\sqrt{c^{-}_u})$ & $k_u$ if $u \le x$ \\
$o(e^u)$ & $e^{k_u 2^{e_u-p}} 2^{e_u+1} k_u$ & $2^{e_u+1} k_u$ if $u \ge x$ \\
$o(\log u)$ & $k_u 2^{1-e_w}$ & \\
\hline
\end{tabular}
\end{center}
\caption{Generic error for several operations, assuming all variables have a
mantissa of $p$ bits, and no overflow/underflow occurs.
The inputs $u$ and $v$ are
approximations of $x$ and $y$ with $|u-x| \le k_u \ulp(u)$ and 
$|v-y| \le k_v \ulp(v)$. The additional rounding error $c_w$ is $1/2$ for
rounding to nearest, and $1$ otherwise.
The value $c^{\pm}_u$ equals $1 \pm k_u 2^{1-p}$.
}
\end{table}
Remark 1: in the addition case, when $u v > 0$,
necessarily one of $2^{e_u-e_w}$ and $2^{e_v-e_w}$ is less than $1/2$,
thus $\err(w)/\ulp(w) \le c_w + {\rm max}(k_u + k_v/2, k_u/2 + k_v)
	\le c_w + \frac{3}{2} {\rm max}(k_u, k_v)$.


\bibliographystyle{acm}
\bibliography{algorithms}

\end{document}
